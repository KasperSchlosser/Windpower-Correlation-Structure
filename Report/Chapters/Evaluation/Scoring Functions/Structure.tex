\section{Consistent Scoring Functions}
\label{evaluation:scores}


As it is not always feasible to determine the entire distribution, point estimates should also be evaluated.

A scoring function is a scoring function that takes a point forecast, e.g. the mean, median, 95\% quantile, etc., and observation, giving a score.

\begin{equation}
    S: \Omega \times \Omega \rightarrow\mathbb{R}
\end{equation}

A consistent scoring function is a scoring function in which the true point estimate, $t$, has the lowest expected score.  \cite{gneitingStrictlyProperScoring2007}

\begin{equation}
    E_{y \sim P}[S(t,y)] \leq E_{y \sim P}[S(x,y)],\quad\forall x \in \Omega
\end{equation}

like with the scoring rules a strictly consistent scoring rule is one where equality only holds for the true point estimate:

\begin{equation}
    E_{y \sim P}[S(t,y)] = E_{y \sim P}[S(x,y)] \iff t = x
\end{equation}

Examples of consistent scoring functions include \gls{mae}, \gls{rmse},\gls{qs}. all three of which is used in this project.


\subsection{Mean Absolute Error}
\label{evaluation:scores:mae}

The \gls{mae} and the \gls{rmse} are two of the more well-known metrics, being defined from distance metrics on $R^n$.

The \gls{mae} is defined by

\begin{equation}
    \text{MAE}(\bar{y}, Y) = \frac{\sum_i^N |y_i - \bar{y}|}{N}
\end{equation}
$Y$ is the observed values and $N$ is the number of observed values.

The \gls{mae} is equally sensitive to each observation, regardless of the other observations.

\begin{equation}
    \frac{d\text{MAE}}{dy_i} = \pm\frac{1}{N}
\end{equation}

\subsection{Root Mean Squared Error}
\label{evaluation:scores:rmse}

Like \gls{mae}, \gls{rmse} is also derived from a distance metric, specifically the Euclidean distance metric.

\begin{equation}
    \text{RMSE}(\bar{y}, Y) = \sqrt{\frac{\sum_i^N (y_i - \bar{y})^2}{N}}
\end{equation}

The \gls{rmse} is not equally sensitive to each observation, unlike the \gls{mae}. Larger observations have a greater impact on the final score. 

\begin{equation}
    \frac{d\text{RMSE}}{dy_i} = \sqrt{\frac{N}{\sum_i^N (y_i - \bar{y})^2}}\frac{2(y_i - \bar{y})}{N} = \frac{1}{\sqrt{\sum_i^N (y_i - \bar{y})^2}} \frac{2(y_i - \bar{y})}{\sqrt{N}}
\end{equation}


\subsection{Quantile Score}
\label{evaluation:scores:qs}

The \gls{qs}, very similar to the \gls{mae}, is calculated as the absolute error from the point forecast, but with a quantile and direction-dependent scaling.

Other names for this score include: Quantile Skill Score, Tick loss, Check loss, Quantile Loss, and Pinball Loss.

The \gls{qs} is defined by

\begin{equation}
    \text{QS}(y, \mu_\tau) = 
    \begin{cases}
    \tau (y - \mu_\tau ) & y \geq \mu_\tau\\
    (1-\tau) (\mu_\tau-y) & y < \mu_\tau
    \end{cases}
    \label{eq:evaluation:qs}
\end{equation}

where $\tau \in [0;1]$ is the quantile and $\mu_\tau$ is the quantile value. \cite{gneitingStrictlyProperScoring2007}

An alternative way to write \cref{eq:evaluation:qs} is as follows:

\begin{equation}
    \text{QS}(\mu_\tau, y) = \max(\tau (y - \mu_\tau),  (\tau-1) (y - \mu_\tau))
\end{equation}

This formulation is used for calculations as many numerical libraries are more optimised for the $\max$ function the an if statement.

In \cref{fig:evaluation:qs} the score is illustrated; we can see that around the 50\% quantile the scores are approximately symmetric. However, for extreme quantiles, only values below the point estimate (above for the upper tail) contribute significantly to the score.

\begin{figure}[ht]
    \centering
    \caption[QS example]{The \gls{qs} of an observation x, for the quantiles 0.1\%, 50\%, and 95\%. Based on the quantiles of the standard normal distribution}
    \includegraphics[width=0.9\linewidth]{Results/Evaluation/Figures/QS.pdf}
    \label{fig:evaluation:qs}
\end{figure}

\subsubsection{QS, MAE \& CRPS}

The \gls{qs} is closely related to two other scores, namely the \gls{mae} and \gls{crps}. 

One can quickly verify that the quantile score for the 50\% quantile is exactly half the \gls{mae}.

\begin{equation}
    QS(\mu_{0.5}, y) = \begin{cases}
    0.5(y - \mu_\tau ) & y \geq \mu_\tau\\
    0.5(\mu_\tau-y) & y < \mu_\tau
    \end{cases} = 0.5 |y- \mu_\tau| = 0.5\text{MAE}(\mu_{0.5},y)
\end{equation}

The relation to \gls{crps} is less obvious, but it can be shown that for \gls{crps} evaluated using simulations \cref{eq:evaluation:crpssim} the \gls{crps} is proportional to the mean quantile loss \cite{brockerEvaluatingRawEnsembles2012}


