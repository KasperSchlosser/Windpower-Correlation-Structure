\section{Variogram Score}
\label{evaluation:vars}

The \gls{vars} is a score for multivariate forecast, and to illustrate the concepts here, two \gls{ar} proccesses, further defined in \cref{autocorrelation} are used. 

\begin{equation}
\begin{split}
    Y_1 := \text{AR}(1): y_t &= 0.8y_{t-1} + \epsilon_t \\
    Y_2 := \text{AR}(2): y_t &= 0.8y_{t-1} - 0.64y_{t-2} + \epsilon_t
\end{split}
\end{equation}

with $\epsilon_t \sim N(0,1),\ \text{\gls{iid}}$

In \cref{fig:evaluation:process} the process response to the intial condition $y_0 = 2$ is illustrated.
The $\text{\gls{ar}}(2)$ process oscillates, while the $\text{\gls{ar}}(1)$ exponentially decays towards 0. 

\begin{figure}[htb]
    \centering\caption[Example processes]{Trajectories for the two example processes with the initial condition $y_0 = 2$. Shaded areas are the 90\% intervals for the processes}
    \includegraphics[width=1\linewidth]{Results/Evaluation/Figures/Process.pdf}
    \label{fig:evaluation:process}
\end{figure}

Every calculated example uses $Y_2$ as the true process, and is calculated from $100.000$ independent simulations.

\subsection{Variogram}

The variogram $\gamma$ of order $p > 0$ for a multivariate prediction $X^T = \begin{bmatrix} x_1, x_2,  \dots \end{bmatrix} $,
is the expected absolute difference, to the power $p$, between all individual elements \cite{scheuererVariogramBasedProperScoring2015}

\begin{equation}
    \gamma_p(F, X)_{i,j} = \expect[F]{\abs{x_i-x_j}^p}
\end{equation}

This matrix is symmetric, and the main diagonal is zero. 

\begin{equation}
    \gamma_p(F,X) = \expect[F]{\begin{pmatrix}
        0 & |x_1 - x_2|^p & |x_1 - x_3|^p & \cdots & |x_1 - x_n|^p \\
        |x_2 - x_1|^p & 0 & |x_2 - x_3|^p & \cdots & |x_2 - x_n|^p \\
        |x_3 - x_1|^p & |x_3 - x_2|^p & 0 & \cdots & |x_3 - x_n|^p \\
        \vdots & \vdots & \vdots & \ddots & \vdots \\
        |x_n - x_1|^p & |x_n - x_2|^p & |x_n - x_3|^p & \cdots & 0 \\
    \end{pmatrix}}
\end{equation}

Due to the absolute value and power $p$, the expected values $\expect{\abs{x_i - x_j}}$ can rarely be theoretically calculated, and simulation usually has to be used \cite{scheuererVariogramBasedProperScoring2015}. 

For the two example processes, their variogram of order $p=0.5$ is shown in \cref{fig:evaluation:variogram}

\begin{figure}[htb]
    \centering
    \caption[VarS Score]{Variograms of order $p = 0.5$ for two the example processes}
    \includegraphics[width=1\linewidth]{Results/Evaluation/Figures/Variogram.pdf}
    \label{fig:evaluation:variogram}
\end{figure}

$Y_1$ shows a steady increase in difference, while $Y_2$ exhibits oscilations

\subsection{VarS}

The \gls{vars} is defined as the (weighted) sum of the squared difference between the expected variogram and the observed variogram of the observation $Y^T = \begin{bmatrix} y_1 & y_2 & \dots & y_n \end{bmatrix}$

\begin{equation}
    VarS_p(F,Y) = \sum_{i,j}^n w_{i,j}\left(\abs{y_i - y_j}^p - \expect[F]{\abs{x_i - x_j}^p} \right)^2
    \label{eq:evaluation:varsorig}
\end{equation}

as the diagonal of the variogram $\abs{x_i - x_i} = 0$ is constant, this is often disregarded $w_{i,i} = 0$.

\subsubsection{Modification to \gls{vars}}

Compared to the normal definition of \gls{vars}, \cref{eq:evaluation:varsorig},  will be used in this project.

As the scores will have a tendency to increases with the size of the variogram $n$, it can be nicer to use a (weighted) mean:

\begin{equation}
    VarS_p(F,Y) = \frac{\sum_{i,j}^n w_{i,j}\left(\abs{y_i - y_j}^p - \expect[F]{\abs{x_i - x_j}^p} \right)^2}{\sum_{i,j}^n w_{i,j}}
\end{equation}

This would, in principle, allow interpretation of scores with different time horizons.

Due to the powers $p$ and $2$ in the calculation interpretation of the score can be difficult. e.g. measurements with a unit $\si{\metre}$ would have a scores with units $\si{\metre^{2p}}$.

To convert back to original units, the $2p$'th root can be taken given a final definition\footnote{For $p=\frac{1}{2}$ this doesn't change the score $2p = 2\frac{1}{2} = 1$.}:

\begin{equation}
    VarS_p(F,Y) = \sqrt[2p]{\frac{\sum_{i,j}^n w_{i,j}\left(\abs{y_i - y_j}^p - \expect[F]{\abs{x_i - x_j}^2} \right)^2}{\sum_{i,j}^n w_{i,j}}}
\end{equation}

For sensible choices of weight, that is $w_{i,j} \geq 0$, the ordering of scores does not change:
\begin{gather}
    \text{\gls{vars}}_1 \geq \text{\gls{vars}}_2 \implies \sqrt[2p]{\frac{\text{\gls{vars}}_1}{\sum_{i,j}^n w_{i,j}}} \geq \sqrt[2p]{\frac{\text{\gls{vars}}_2}{\sum_{i,j}^n w_{i,j}}}
\end{gather}


\subsection{Order}

The order $p$ determines what observations are emphasised. higher orders prioritising larger differences. The higher orders will in general have more variance, as can be seen in \cref{fig:evaluation:varsdist}

\begin{figure}[htb]
    \centering
    \caption[Scores Distribution]{Distribution of scores for the example \gls{ar}(2) process.}
    \includegraphics[width=1\linewidth]{Results/Evaluation/Figures/Vars dist.pdf}

    \label{fig:evaluation:varsdist}
\end{figure}

Different situations can be constructed where one order outperforms the other. its not possible to choose one ideal order.
Here however the order $p =0.5$ will be used exclusively.

\subsection{Weighting}

The weighting $w$ can be chosen in a variety of ways. \textcite{scheuererVariogramBasedProperScoring2015}. Shows that using appropriate weights can improve the signal-to-noise ratio. A pragmatic weighting is suggest of the inverse distance between between elements

\begin{equation}
    w_i,j =\frac{1}{\abs{i-j}}
    \label{eq:evaluation:weightschuer}
\end{equation}

This weighting was also used by \textcite{jorgensenSequentialMethodsError2025}, and suggested in \textcite{bjerregardIntroductionMultivariateProbabilistic2021}. The idea being that correlation tend to decrease with distance contributing only noise.

This weighting primarily points weight on the observations around the main diagonal, as can be seen in \cref{fig:evaluation:varsweight}

\begin{figure}[htb]
    \centering
    \caption[VarS Weight]{Weight as a percentage of the maximum weight}
    \includegraphics[width=0.5\linewidth]{Results/Evaluation/Figures/Weight.pdf}
    \label{fig:evaluation:varsweight}
\end{figure}

For the example processes there is still information left, even at the larger lags. Additional there is a difference in stationary variance, can be seen in \cref{fig:evaluation:process}, which would only be evident at larger lags.

Weighting with the inverse distance completely disregards this information. \cref{fig:evaluation:scorediff} shows the expected difference in score $\expect{\text{\gls{vars}}(Y_2, Y_1) - \text{\gls{vars}}}(Y_1, Y_1)$ at each position.

For the equal weight the contributions at the tails are around 5\% the size of the largest contribution, while for the inverse distance weighting There is no contribution outside lag 6.

\begin{figure}[htb]
    \centering
    \caption[Expected Score Difference]{Expected difference in score as a percentage of max max score. for the two example processes. Left: Equal Weighting, Right: Inverse Distance Weighting}
    \includegraphics[width=1\linewidth]{Results/Evaluation/Figures/score diff.pdf}
    \label{fig:evaluation:scorediff}
\end{figure}

This results and scores depending primarily the main diagonal, increasing the variance, as can be seen in \cref{fig:evaluation:diffdist}.

This decreases the ability of the score to differentiate the two processes, as given in \cref{tab:evaluation:precision}.

\begin{table}[htb]
\centering
\caption[Precision by Weight]{Precision of the \gls{vars}, for the equally weighted score and the score weighted with inverse distance}
\label{tab:evaluation:precision}
\begin{tabular}{lrr}
\toprule
Weight & Equal & Inverse Distance \\
\midrule
Score Difference & 0.82 & 0.77 \\
\bottomrule
\end{tabular}
\end{table}


\begin{figure}[htb]
    \centering
    \caption[Distribution of Score Differences]{Distribution of score difference $\expect{\text{\gls{vars}}(Y_2, Y_1) - \text{\gls{vars}}(Y_1, Y_1)}$}
    \includegraphics[width=1\linewidth]{Results/Evaluation/Figures/diff dist.pdf}
    \label{fig:evaluation:diffdist}
\end{figure}

Due to this it is decided to use an equal weighting for \gls{vars} calculations
\begin{equation}
    w_{i,j} = \begin{cases}
        1 & i \neq j \\
        0 & i = j
    \end{cases}
\end{equation}