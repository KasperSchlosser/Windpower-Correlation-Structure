\section{Proper Scoring Rules}
\label{evaluation:properscores}

A scoring rule is any functional that takes an estimated distribution, along with an observed outcome, and calculates a score\cite{gneitingStrictlyProperScoring2007}.

According to \textcite{gneitingStrictlyProperScoring2007} a scoring rule is a functional 
\begin{equation}
    S: P \times \Omega \to \mathbb{R}
\end{equation}
Where $P$ a is a set of probability measures over the outcome space $\Omega$.

A proper scoring rule, is any scoring rule where the true distributions have the lowest expected score.

That is defining the expected score
\begin{equation}
    S(P,Q) = \mathbb{E}_{Q}[S(P,y)] = \int_\Omega S(P,\omega)dQ(\omega)
    \label{eq:evaluation:properineq}
\end{equation}

a proper scoring rule fulfils 
\begin{equation}
    S(Q,Q) \leq S(P,Q)    
\end{equation}

Where $Q$, $P$ are probability measures over $\Omega$ \cite{gneitingStrictlyProperScoring2007}

A scoring rule is strictly proper if the equality in \cref{eq:evaluation:properineq} only holds for the true distribution.

\begin{equation}
    S(Q,Q) = S(P,Q) \iff P = Q
\end{equation}

For this projekt two scoring functions will be used, the \gls{crps}, which is strictly proper, and the \gls{vars} which is proper. 

\subsubsection{Orientation}

Scores can be defined in both positive and negative orientation, considering the following equation.

\begin{equation}
    S(P,\omega) > S(Q,\omega)
    \label{eq:evaluation:orientation}
\end{equation}

In positive orientation, higher scores are better, that is, \cref{eq:evaluation:orientation} indicates that P is better than Q.

With negative orientation, lower scores are better; here \cref{eq:evaluation:orientation} would instead indicate that Q is the better model.

In \textcite{gneitingStrictlyProperScoring2007} positive orientation is primarily used. Here, the negative orientation is used exclusively.

\subsubsection{Averaging Scores}
As most situations would require evaluating on more than a single observation, a method of aggrating scores is needed.

The obvious choice is to take the mean.

\begin{equation}
    S(P,\{y_i\}) = \frac{\sum^N_i S(P,y_i)}{N}
\end{equation}
or potentially a weighted mean.

\subsection{Continuous Ranked Probability Score}
\label{evaluation:crps}

The \gls{crps} is a strictly proper scoring rule. The base score is calculated as the integral of the absolute squared difference between the base distribution and the "observed" distribution of the point x.

\gls{crps} is defined by:
\begin{equation}
    \text{CRPS}(F,y) =  \int_{-\infty}^{\infty} (F(x) - \mathbb{I}_{x \leq y}(x))^2 dx = \int_{-\infty}^y F(x)^2 dx+ \int_y^\infty(F(x)-1)^2dx
\end{equation}

The calculation of \gls{crps} is visualised in \cref{fig:evaluation:crps-obs}.

\begin{figure}[htb]
    \centering
    \caption[CRPS Observed Distribution"]{The observed distribution for a data point, here $x = 0.5$. The resulting \gls{crps} scores is the colored area}
    \includegraphics[width=1\linewidth]{Results/Evaluation/Figures/CRPS observation.pdf}
    \label{fig:evaluation:crps-obs}
\end{figure}

There are two difficulties in calculating the \gls{crps}:

\begin{enumerate}
    \item The distribution $F$ does not always exist in closed form.
    \item The integral can be intractable to calculate even numerically.
\end{enumerate}

As a solution two of these problems, \gls{crps} can also be formulated as\cite{gneitingStrictlyProperScoring2007}:

\begin{equation}
    CRPS(F,y) = \expect{|x_1-y|} - \frac{1}{2} \expect{|x_1-x_2|}
    \label{eq:evaluation:crpssim}
\end{equation}

where $x_1, x_2 \sim F$ and \gls{iid}.

This formulation allows calculation of \gls{crps} from simulation, as both terms $\expect[F]{|X_1-y|}$ and $\expect[F]{|X_1-X_2|}$ can be estimated by sampling from $F$.
