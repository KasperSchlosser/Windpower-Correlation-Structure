\subsection{Kullback-Liebler Divergence}
\label{evaluation:distribution:kullback}

\gls{kl} is a another metric of "distance" between two distributions.\gls{kl} is defined by \cite{lovricInternationalEncyclopediaStatistical2011}:

\begin{equation}
    D_{KL}(P||Q) = \int_\Omega p(x) \log\left(\frac{p(x)}{q(x)}\right) dx
    \label{eq:evaluation:kl}
\end{equation}

One requirement here is that $q(x)$ has the same support as $p(x)$ or the integral becomes undefined.

\gls{kl} is not symmetric, the direction of the scores is therefore important. Here the the divergence in \cref{eq:evaluation:kl} will be referred to as the divergence of $Q$ under $P$.

Interpretation of the \gls{kl} is the expected excess information, or excess entropy, by assuming distribution $Q$ when the true distribution is $P$\cite{lovricInternationalEncyclopediaStatistical2011}.

Defining information as:

\begin{equation}
    I_F(x) = -\ln(f(x))
\end{equation}

\cref{eq:evaluation:kl} can be written as

\begin{equation}
    D_{KL}(P||Q) = \int_\Omega p(x) (I_Q(x) - I_P(x)) dx = \expect[P]{I_Q(x)} - \expect[P]{I_P(x)}
\end{equation}

For the two example distributions, a graph the integrands of the \gls{kl} can be seen in \cref{fig:evaluation:klscore}, the divergence for $G$ under $F$ becomes rather larger in the right tail, while the divergence of $F$ under $G$ is relatively small over the entire distribution.

\begin{figure}[htb]
    \centering
    \caption[KL-Divergence]{Expected excess information for the two example distributions}
    \includegraphics[width=1\linewidth]{Results/Evaluation/Figures/KL-Divergence.pdf}
    \label{fig:evaluation:klscore}
\end{figure}

The actual calculated divergences can be see in \cref{tab:evaluation:kl}
\begin{table}[ht]
\centering
\caption[Kullback-Leibler Divergence]{Kullback-Leibler divergence of the two example distributions.}
\label{tab:evaluation:kl}
\begin{tabular}{rr}
\toprule
$KL(F||G)$ & $KL(G||F)$ \\
\midrule
1.64 & 0.49 \\
\bottomrule
\end{tabular}
\end{table}


As noted the divergence of $G$ under $F$ is larger than the reverse, by about a factor 3. The observation we would get from $F$ would be quite unlikely according to $G$. 







