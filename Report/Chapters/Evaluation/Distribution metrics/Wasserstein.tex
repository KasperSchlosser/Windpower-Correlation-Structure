\subsection{Wasserstein Metric}
\label{evaluation:distribution:wasserstein}


The Wasserstein metric is a distance measure between two distributions. The metric arises through an optimal transportation problem. \cite{kolouri2017a}

For distributions P and Q the Wasserstein metric of order p is optimal transport with cost $|x-y|^2$. 

\begin{equation}
    W_p(P,Q) = \min_{\gamma}\int_{\Omega \times \Omega} |x-y|^p d\gamma(x,y)
\end{equation}

More simply, the Wasserstein metric is the least amount of probablity that has to be move to transform the distribution P into the distribution Q.\cite{kolouri2017a}

For one-dimensional distributions, the definition luckily simplifies to the $l_p$ distance between the quantile functions.
\begin{equation}
    W_p(P,Q) = \int_0^1 |P^{-1}(x) - Q^{-1}(x)|^p dx
\end{equation}

The Wasserstein metric is a metric for $p \in [1;\infty]$ for $p < 1$ the function no longer satisfies the triangle inequality and thus not a metric. However, it can still be used as a score.

For the two example distributions the Wasserstein distance between them can be seen in \cref{tab:evaluation:wasserstein}.

\begin{table}[ht]
\centering
\caption[Wasserstein Distance]{Wasserstein distance for the two example distributions.}
\label{tab:evaluation:wasserstein}
\begin{tabular}{lrrrr}
\toprule
 & \multicolumn{4}{c}{p} \\
 & 0.1 & 0.5 & 1.0 & 2.0 \\
\midrule
$W_p$ & 0.57 & 0.68 & 0.79 & 0.95 \\
\bottomrule
\end{tabular}
\end{table}


The order, $p$ determines what is emphasised by the score, in the limit score is integral of the indicator function 

\begin{equation}
    \lim_{p \to 0}W_p(P,Q) = \int_0^1 \mathbb{I}_{P^{-1}(x) \neq Q^{-1}(x)}(x) dx
\end{equation}

While for higher orders only the larger difference contribute to the score. an illustration can be seen in \cref{fig:evaluation:wasserstein-order}

\begin{figure}[htb]
    \centering
    \caption[Wasserstein Order]{The Wasserstein score's order, p, prioritizes various distances, highlighting larger discrepancies more significantly.}
    \includegraphics[width=1\linewidth]{Results/Evaluation/Figures/Wasserstein Distance.pdf}
    \label{fig:evaluation:wasserstein-order}
\end{figure}







