\section{Distribution of Power Production}

An example of the estimated distribution for the feature model can be seen in \cref{fig:basis:feature-example}. Examples for all models can be seen in \cref{appendix:basis:examples}

\begin{figure}[htb]
    \centering
    \caption[Feature Example]{Example of the marginal distribution from the feature model forecast. Shaded area is the 90\% confidence interval}
    \includegraphics[width=0.9\linewidth]{Results/Basis/Figures/Feature example.pdf}
    \label{fig:basis:feature-example}
\end{figure}

In general, the feature, \gls{nabqr} and \gls{taqr} models seem to predict similarly, with the \gls{taqr} model giving narrower intervals at times.

This could indicate that the dynamics of the system has changed over time and that the \gls{taqr} model is able to handle this.

The ensemble forecasts are markedly narrower.

while the ensembles seem to follow the general structure of the observations. It is clear that they do not represent the full distribution of possible outcomes very well; see \cref{fig:basis:comparison}.

\begin{figure}[htb]
    \centering
    \caption[Feature-Ensemble Comparison]{Example of the marginal forecast from the feature model. Shaded area is the 90\% confidence interval}
    \includegraphics[width=0.9\linewidth]{Results/Basis/Figures/Comparison.pdf}
    \label{fig:basis:comparison}
\end{figure}

\newpage
\subsection{Scores}

For \gls{mae} and \gls{rmse} the estimated median was used as the point forecast.

For \gls{crps} and \gls{vars}, 1000 samples were generated from the distributions and used to calculate the scores.

The resulting scores can be seen in \cref{tab:basis:scores}. 

\begin{table}[htb]
\centering
\caption[Marginal Distribution Scores]{Scores for the estimated marginal distributions. Bold scores indicated the minimum scores for the the zone and measure. Scores coloured blue are within 10\% of the minimum score}
\label{tab:basis:scores}
\begin{tabular}{llrrrr}
\toprule
 &  & MAE & RMSE & CRPS & VarS \\
 & Model &  &  &  &  \\
\midrule
\multirow[r]{6}{*}{DK1-offshore} & Ensemble & {\cellcolor[HTML]{269BE3}} 127.79 & {\cellcolor[HTML]{269BE3}} 184.71 & 99.71 & \bfseries \itshape {\cellcolor[HTML]{269BE3}} 30.09 \\
 & Ensemble - Raw & {\cellcolor[HTML]{269BE3}} 127.57 & {\cellcolor[HTML]{269BE3}} 182.66 & 100.78 & 34.00 \\
 & Feature & {\cellcolor[HTML]{269BE3}} 124.55 & {\cellcolor[HTML]{269BE3}} 183.81 & {\cellcolor[HTML]{269BE3}} 90.21 & 34.65 \\
 & NABQR & \bfseries \itshape {\cellcolor[HTML]{269BE3}} 122.75 & \bfseries \itshape {\cellcolor[HTML]{269BE3}} 179.60 & {\cellcolor[HTML]{269BE3}} 89.47 & 35.64 \\
 & Simple & {\cellcolor[HTML]{269BE3}} 128.85 & {\cellcolor[HTML]{269BE3}} 185.41 & {\cellcolor[HTML]{269BE3}} 92.47 & 34.74 \\
 & TAQR & {\cellcolor[HTML]{269BE3}} 123.51 & {\cellcolor[HTML]{269BE3}} 181.40 & \bfseries \itshape {\cellcolor[HTML]{269BE3}} 89.26 & 36.45 \\
\hline
\multirow[r]{6}{*}{DK1-onshore} & Ensemble & {\cellcolor[HTML]{269BE3}} 197.40 & \bfseries \itshape {\cellcolor[HTML]{269BE3}} 269.89 & {\cellcolor[HTML]{269BE3}} 145.62 & {\cellcolor[HTML]{269BE3}} 42.11 \\
 & Ensemble - Raw & {\cellcolor[HTML]{269BE3}} 199.37 & {\cellcolor[HTML]{269BE3}} 270.60 & {\cellcolor[HTML]{269BE3}} 145.71 & \bfseries \itshape {\cellcolor[HTML]{269BE3}} 41.97 \\
 & Feature & {\cellcolor[HTML]{269BE3}} 201.68 & {\cellcolor[HTML]{269BE3}} 282.26 & {\cellcolor[HTML]{269BE3}} 142.00 & 49.17 \\
 & NABQR & {\cellcolor[HTML]{269BE3}} 213.63 & 301.31 & 154.09 & 50.19 \\
 & Simple & {\cellcolor[HTML]{269BE3}} 201.41 & {\cellcolor[HTML]{269BE3}} 280.18 & {\cellcolor[HTML]{269BE3}} 144.29 & 50.16 \\
 & TAQR & \bfseries \itshape {\cellcolor[HTML]{269BE3}} 195.57 & {\cellcolor[HTML]{269BE3}} 274.22 & \bfseries \itshape {\cellcolor[HTML]{269BE3}} 139.97 & 48.92 \\
\hline
\multirow[r]{6}{*}{DK2-offshore} & Ensemble & 161.87 & 246.89 & 127.12 & 35.51 \\
 & Ensemble - Raw & 156.16 & 236.59 & 127.85 & 35.92 \\
 & Feature & \bfseries \itshape {\cellcolor[HTML]{269BE3}} 104.54 & {\cellcolor[HTML]{269BE3}} 167.05 & \bfseries \itshape {\cellcolor[HTML]{269BE3}} 77.55 & \bfseries \itshape {\cellcolor[HTML]{269BE3}} 31.02 \\
 & NABQR & {\cellcolor[HTML]{269BE3}} 114.10 & {\cellcolor[HTML]{269BE3}} 177.67 & {\cellcolor[HTML]{269BE3}} 83.69 & {\cellcolor[HTML]{269BE3}} 34.00 \\
 & Simple & 152.64 & 234.26 & 115.38 & 42.82 \\
 & TAQR & {\cellcolor[HTML]{269BE3}} 104.82 & \bfseries \itshape {\cellcolor[HTML]{269BE3}} 164.67 & {\cellcolor[HTML]{269BE3}} 77.63 & {\cellcolor[HTML]{269BE3}} 34.05 \\
\hline
\multirow[r]{6}{*}{DK2-onshore} & Ensemble & 72.40 & 100.67 & 53.75 & 15.89 \\
 & Ensemble - Raw & 76.85 & 108.45 & 59.91 & 11.84 \\
 & Feature & {\cellcolor[HTML]{269BE3}} 43.60 & {\cellcolor[HTML]{269BE3}} 59.48 & {\cellcolor[HTML]{269BE3}} 30.54 & \bfseries \itshape {\cellcolor[HTML]{269BE3}} 9.85 \\
 & NABQR & {\cellcolor[HTML]{269BE3}} 42.19 & {\cellcolor[HTML]{269BE3}} 57.04 & {\cellcolor[HTML]{269BE3}} 30.17 & {\cellcolor[HTML]{269BE3}} 10.19 \\
 & Simple & {\cellcolor[HTML]{269BE3}} 43.60 & {\cellcolor[HTML]{269BE3}} 60.08 & {\cellcolor[HTML]{269BE3}} 31.59 & {\cellcolor[HTML]{269BE3}} 10.29 \\
 & TAQR & \bfseries \itshape {\cellcolor[HTML]{269BE3}} 40.20 & \bfseries \itshape {\cellcolor[HTML]{269BE3}} 55.74 & \bfseries \itshape {\cellcolor[HTML]{269BE3}} 28.79 & {\cellcolor[HTML]{269BE3}} 9.87 \\
\hline
\multirow[r]{6}{*}{Geometric mean score} & Ensemble & 131.12 & 187.62 & 99.80 & {\cellcolor[HTML]{269BE3}} 29.08 \\
 & Ensemble - Raw & 132.18 & 188.71 & 102.98 & {\cellcolor[HTML]{269BE3}} 27.91 \\
 & Feature & {\cellcolor[HTML]{269BE3}} 103.44 & {\cellcolor[HTML]{269BE3}} 150.68 & {\cellcolor[HTML]{269BE3}} 74.22 & \bfseries \itshape {\cellcolor[HTML]{269BE3}} 26.86 \\
 & NABQR & {\cellcolor[HTML]{269BE3}} 106.00 & {\cellcolor[HTML]{269BE3}} 153.03 & {\cellcolor[HTML]{269BE3}} 76.81 & {\cellcolor[HTML]{269BE3}} 28.06 \\
 & Simple & 114.64 & 164.44 & 83.51 & 29.60 \\
 & TAQR & \bfseries \itshape {\cellcolor[HTML]{269BE3}} 100.44 & \bfseries \itshape {\cellcolor[HTML]{269BE3}} 146.18 & \bfseries \itshape {\cellcolor[HTML]{269BE3}} 72.69 & {\cellcolor[HTML]{269BE3}} 27.82 \\
\bottomrule
\end{tabular}
\end{table}


For DK1, the scores are generally close, except for \gls{vars}, where the ensemble forecasts are better. 
The better\gls{vars} scores indicate that the ensemble forecast captures the dynamics of the system. 


The tendency of \gls{nabqr} to overfit is also evident. Even with the early stopping criteria used to control for it. In DK1-onshore \gls{nabqr} performed worse than all other models in all scores.

For the simple model, it seems to perform comparably with the more complex models, except in DK2-offshore.

As speculated earlier, this is probably caused by the smaller input, causing the model to have no information when the base ensembles collapse.

As the feature model performs well in all situations, it was chosen as a basis for further analysis of autocorrelation. 

while the \gls{taqr} model was probably preferable, having better scores in most cases. The computational problems meant that there would be little training data for further analysis. 

\subsection{Residual Structure}

Using the feature model as a basis for the marginal distribution, we can find the pseudo-residuals; an example for DK1-onshore can be found in \cref{fig:basis:feature-example}, and for all zones in \cref{appendix:basis:pseudo}.

The model seem to capture the main distribution quite well, but the tail distribution is not as well described.

For DK1-offshore there are especially problems in the lower tail and some large outliers.

With regard to correlation, we can see from the \gls{acf} and \gls{pacf} graphs that there is a significant autocorrelation at lag 1 and possibly lag 2. 

There is some indication of a correlation at lag 24, showing "humps" around this lag in \gls{acf}.

\begin{figure}[htb]
    \centering
    \caption[DK1-onshore - Feature Residuals]{Pseudo-residuals for the feature model in DK1-onshore.}
    \includegraphics[width=1\linewidth]{Results/Basis/Figures/Residuals/DK1-onshore.pdf}
    \label{fig:basis:feature-residuals}
\end{figure}

