\chapter{Pseudo-Residuals}
\label{Pseudo}
\addcontentsline{lof}{chapter}{Pseudo-residuals}

As the main idea for this thesis relies on the concept of pseudo-residuals, these will be introduced in this chapter.

For random, possibly multivariate, variables $y_i$ following a distribution F, there are two distinct pseudo-residuals associated with this variable: the uniform pseudo-residual, $u-i$ and the normal pseudo-residual $z_i$. \cite{zucchiniHiddenMarkovModels2017}

The uniform pseudo-residual is defined as the \gls{cdf} of the variable, while the normal pseudo-residual is the quantile of the standard normal distribution corresponding to the uniform pseudo-residual.
\begin{align}
    y_i &\sim F \\
    u_i &:= F(y_i) \\
    z_i &:= \Phi^{-1}(u_i)
\end{align}
where $\Phi(x)$ is the \gls{cdf} of the standard normal distribution. It can quickly be seen that the uniform pseudo-residual follows a uniform distribution on $[0;1]$
\begin{equation}
\begin{gathered}
    P(u_i \leq t) = P(y_i \leq F^{-1}(t)) = t \\
    \implies \\ 
    u_i \sim U_{[0;1]}(t)
\end{gathered}
\end{equation}

and by extension the normal pseudo-residuals will follow a standard normal distribution. \cite{zucchiniHiddenMarkovModels2017}
\begin{equation}
    z_i \sim N(0,1^2)
\end{equation}

In this thesis, the term pseudo-residual will refer to the normal pseudo-residual, with the uniform specified as necessary.

\newpage
\section{Model Checking}
The first use case for pseudo-residuals is model checking. Test and heuristics for verifying a specific distribution can be hard to find. diagnostics of a model, in which each observation might follow completely different distributions, can be impossible.

By contrast, tests and heuristics for verifying a normal distribution are well-known and plentiful, being taught in most introductory statistics courses, if slightly divisive w.r.t. actual normality tests.

Pseudo-residuals can be used for model verification. If the model correctly specifies the true distribution, then the pseudo-residuals will follow a normal distribution.

In this project, six plots will be used for model diagnostics; no tests are performed.

\begin{itemize}
    \item Histogram of normal pseudo-residuals
    \item Histrogram of uniform pseudo-residuals
    \item \gls{qq}-plot of normal pseudor-residuals
    \item Outlier plot of normal pseudo-residuals
    \item \gls{acf} plot of normal pseudo-residuals
    \item \gls{pacf} plot of normal pseudo-residuals
\end{itemize}

To illustrate the methods an example is made. Two distributions are chosen, a standard normal distribution and a students t-distribution with two \gls{df}, shifted to a mean of 0.2.

\begin{equation}
\begin{split}
    F_1 = N(0,1^2) \\
    F_2 = t(2,\mu = 0.2)
\end{split}
\end{equation}

1000 samples were drawn from the first distribution $x_i \sim F_1$. Diagnostic plots can be seen in \cref{fig:pseudo:correctmodel,fig:pseudo:wrongmodel} for the correctly specified and misspecified models, respectively.

\subsection{Histograms and \gls{qq}-plot}
The histograms, the two middle plots of \cref{fig:pseudo:correctmodel,fig:pseudo:wrongmodel}, and the \gls{qq} plot, the bottom right plot, gives a visual indication of the fit of the model.

The uniform histogram should be a horizontal line, the normal histogram should follow the well-known bell-shaped curve, and the \gls{qq}-plot should be a straight line. \cite{zucchiniHiddenMarkovModels2017}

\subsection{Outlier plot}

The outlier plot, bottom left in \cref{fig:pseudo:correctmodel,fig:pseudo:wrongmodel}, is the normal pseudo-residuals plotted as a function of row/index with two sets of limits plotted. 

Firstly this allows for for visual inspection of an error structure, as well as giving an indication of possible outliers.

For a normal distribution 95\% of all observations are expected to lie within $\pm 1.96 = \Phi^{-1}(1 - \frac{0.05}{2})$, the green lines.
if too many or few observations lie outside these bounds, it would indicate a model problem.

Outliers are observations that are very unlikely according to the distribution, that is, observations that cannot be explained adequately by the model.
The red limits of the can be taken as a guide for classifying outliers. Calculated as the 95\% interval of a standard normal distribution, the Bonferroni corrected for multiple comparisons. 
\begin{equation}
    \text{Outlier limit} = \pm \Phi^{-1}\left(1-\frac{0.05}{2 N}\right)
\end{equation}

where $N$ is the number of observations.

In the example there are no observations outside the limits,

\subsection{\gls{acf} and \gls{pacf}}

The \gls{acf}, top left in \cref{fig:pseudo:correctmodel,fig:pseudo:wrongmodel}, shows the autocorrelation of the pseudo-residuals. The autocorrelation is defined as the correlation between pseudo-residuals at lag k. \cite{shumwayTimeSeriesAnalysis2017}

\begin{equation}
    \text{\gls{acf}}(k) = \text{corr}(z_i,z_{i+k})
\end{equation}

Similarly, the \gls{pacf}, top right plot of \cref{fig:pseudo:correctmodel,fig:pseudo:wrongmodel}, defines the correlation of pseudo-residuals with the linear dependence of previous observations removed\cite{shumwayTimeSeriesAnalysis2017}

\begin{equation}
    \text{\gls{pacf}(k)} = \text{corr}(z_i, z_{i+k} | z_{i+1}, z_{i+2}, \dots ,z_{i+k-1})
\end{equation}

For these plots to make sense, there has to be some structure, spatial, temporal, etc. to the index. In the example there is no autocorrelation, the samples were drawn independently.

\begin{figure}[htb]
    \centering
    \caption[Pseudoresiduals - Correct model]{Diagnostic plots for the Pseudoresiduals of the correctly specified model $y \sim N(0,1^2)$}
    \includegraphics[width=1\linewidth]{Results/Pseudoresiduals/Figures/normal resid.pdf}
    \label{fig:pseudo:correctmodel}
\end{figure}

\begin{figure}[htb]
    \centering
    \caption[Pseudoresiduals - Mispecified model]{Diagnostic plots for the Pseudoresiduals of the mispecified model $y \sim t(2, \mu = 0.2)$}
    \includegraphics[width=1\linewidth]{Results/Pseudoresiduals/Figures/t resid.pdf}
    \label{fig:pseudo:wrongmodel}
\end{figure}
\clearpage
\section{Modelling Correlation}
\label{psudo:copula}

The second use case is in modelling the correlation structure, which is explored in detail in \cref{autocorrelation}. 

The correlation structure of a multivariate variable $Y^T = \begin{bmatrix} y_1, y_2, \dots \end{bmatrix}$ describes how the individual random variables are distributed in relation to each other, that is, the joint distribution $F(Y) = F(y_1,y_2,\dots)$.\cite{madsenTimeSeriesAnalysis2007}

The marginal distribution of an individual element $F_{y_i}$ is the distribution over all possible outcomes\cite{madsenTimeSeriesAnalysis2007}:
\begin{equation}
\begin{split}
    f{y_1} &= \int\dots\int f(y_1,y_2,\dots,y_n)d{y_2}\dots dy_n \\
    F_{y_1} &= \int f_{y_1}(x) dx
\end{split}
\end{equation}

The most well-known multivariate distribution is probably the multivariate normal, where all marginal distributions are also (multivariate) normal\cite{madsenTimeSeriesAnalysis2007}. 

Other multivariate distributions exist, though not as numerous as univariate distributions. It can be difficult to describe the correlation between random variables with different marginal distributions\cite{huber2015a}

For time-varying process, different methods exist to describe the correlation structure. In discrete time, \gls{arima} and related models are widely used \cite{shumwayTimeSeriesAnalysis2017}. In continuous time, the study of stochastic differential equations can be mentioned\cite{thygesenStochasticDifferentialEquations2023}.

\subsection{Copulas}

For describing multivariate distributions with a variety of marginals, the theory of copulas was developed. Formally a copula $C: [0;1]^d \to [0;1]$ is a multivariate distribution with uniform marginals, $U_[0;1]$\cite{ruppertCopulas2015}.

Sklar's Theorem states that every multivariate distribution $F(y_1,y_2, \dots, y_n)$ with marginals $F_1, F_2, \dots, F_n$ can be expressed as a copula\cite{ruppertCopulas2015}:
\begin{equation}
    F(y_1,y_2, \dots, y_n) = C\left(F_1(y_1), F_2(y_2), \dots, F_n(y_n)\right) 
\end{equation}
and for the density
\begin{equation}
    f(y_1,y_2, \dots, y_n) = f_1(y_1) f_2(y_2)\dots f_n(y_n)\ c\left(F_1(y_1), F_2(y_2), \dots, F_n(y_n)\right)
\end{equation}

The copulas provide a general way to describe multivariate distributions.
An example of a copula is the Gaussian copula, where the correlation structure is described by a multivariate normal distribution\cite{ruppertCopulas2015}.

\begin{equation}
\begin{split}
    F(y_1, y_2, \dots, y_n) &= C_{gauss}(F_1(y_1), F_2(y_2), \dots, F_n(y_n))\\
    &= \Phi_\Sigma\left(\Phi_\Sigma^{-1}(F_1(y_1)), \Phi_\Sigma^{-1}(F_2(y_2)), \dots, \Phi_\Sigma^{-1}(F_n(y_n))\right) 
\end{split}
\end{equation}

where $\Phi_\Sigma = MVN(0, \Sigma)$ is a is \gls{cdf} of a multivariate normal distribution with covariance matrix $\Sigma$.




% For a random variable $X_i \sim F$, 
% we can define the uniform \gls{pseudo}, $U_i$ as:
% \begin{equation*}
% U_i = F(X_i) 
% \end{equation*}
% this implies
% \begin{equation*}
%     U_i \sim U(0,1)
% \end{equation*}

% Where F is the \gls{cdf} of $X_i$ and $U(0,1)$ is the uniform distribution with limits 0 and 1. Further, define the normal \gls{pseudo} as 
% \[
% Z_i = \bm{\Phi}^{-1}(U_i) \implies Z_i \sim N(0,1)
% \]
% This gives three ways of presenting the random variable $x_i$\cite{hmmbook}

% \begin{enumerate}
%     \item $x_i \sim F$: Original space
%     \item $u_i \sim U(0,1)$: \gls{cdf}-space
%     \item $z_i \sim N(0,1)$: Normal-space
% \end{enumerate}

% \subsection{Model checking with pseudo residuals}

% If the random variable $x_i$ follows the distribution $F$ then both $u_i$ and $z_i$ follows a uniform- and standard normal-distribution.\\
% Standard diagnostic tools, suited for the these distributions, can be used for checking distribution assumption. \cite{hmmbook}

% some tools we will use are 

% \subsubsection{Histograms}

% Both the uniform-and normal distribution have distinct shapes
% \begin{figure}[H]
%     \centering
%     \begin{subfigure}{0.5\linewidth}
%     \centering
%     \includegraphics[width=\linewidth]{Figures/Autocorrelation/model checking/pwl_cdfdist.png}
%     \caption{uniform}
%     \label{asdasdasdasd}
%     \end{subfigure}%
%     \begin{subfigure}{0.5\linewidth}
%     \centering
%     \includegraphics[width=\linewidth]{Figures/Autocorrelation/model checking/pwl_normaldist.png}
%     \caption{normal}
%     \label{asdasdasdasdsadasd}
%     \end{subfigure}
    
% \end{figure}

% \subsubsection{QQ-plot}
% \begin{figure}[H]
%     \centering
%     \includegraphics[width=0.8\linewidth]{Figures/Autocorrelation//model checking/DK2-onshore_ARMAtest_after_probplot.png}
%     \caption{Enter Caption}
%     \label{fig:enter-label}
% \end{figure}

% \subsubsection{Outliers}

% \begin{figure}[H]
%     \centering
%     \includegraphics[width=0.8\linewidth]{Figures/Autocorrelation/model checking/DK2-onshore_SARMAtest_after_scatter.png}
%     \caption{outliers}
%     \label{fig:enter-label}
% \end{figure}

% \subsubsection{\gls{acf} \& \gls{pacf}}

% \begin{figure}[H]
%     \centering
%     \includegraphics[width=0.5\linewidth]{Figures/Autocorrelation/model checking/Constant_autocorrelation.png}
%     \caption{acf}
%     \label{fig:enter-label}
% \end{figure}




