\section{Model Checking}
The first use case for pseudo-residuals is model checking. Test and heuristics for verifying a specific distribution can be hard to find. diagnostics of a model, in which each observation might follow completely different distributions, can be impossible.

By contrast, tests and heuristics for verifying a normal distribution are well-known and plentiful, being taught in most introductory statistics courses, if slightly divisive w.r.t. actual normality tests.

Pseudo-residuals can be used for model verification. If the model correctly specifies the true distribution, then the pseudo-residuals will follow a normal distribution.

In this project, six plots will be used for model diagnostics; no tests are performed.

\begin{itemize}
    \item Histogram of normal pseudo-residuals
    \item Histrogram of uniform pseudo-residuals
    \item \gls{qq}-plot of normal pseudor-residuals
    \item Outlier plot of normal pseudo-residuals
    \item \gls{acf} plot of normal pseudo-residuals
    \item \gls{pacf} plot of normal pseudo-residuals
\end{itemize}

To illustrate the methods an example is made. Two distributions are chosen, a standard normal distribution and a students t-distribution with two \gls{df}, shifted to a mean of 0.2.

\begin{equation}
\begin{split}
    F_1 = N(0,1^2) \\
    F_2 = t(2,\mu = 0.2)
\end{split}
\end{equation}

1000 samples were drawn from the first distribution $x_i \sim F_1$. Diagnostic plots can be seen in \cref{fig:pseudo:correctmodel,fig:pseudo:wrongmodel} for the correctly specified and misspecified models, respectively.

\subsection{Histograms and \gls{qq}-plot}
The histograms, the two middle plots of \cref{fig:pseudo:correctmodel,fig:pseudo:wrongmodel}, and the \gls{qq} plot, the bottom right plot, gives a visual indication of the fit of the model.

The uniform histogram should be a horizontal line, the normal histogram should follow the well-known bell-shaped curve, and the \gls{qq}-plot should be a straight line. \cite{zucchiniHiddenMarkovModels2017}

\subsection{Outlier plot}

The outlier plot, bottom left in \cref{fig:pseudo:correctmodel,fig:pseudo:wrongmodel}, is the normal pseudo-residuals plotted as a function of row/index with two sets of limits plotted. 

Firstly this allows for for visual inspection of an error structure, as well as giving an indication of possible outliers.

For a normal distribution 95\% of all observations are expected to lie within $\pm 1.96 = \Phi^{-1}(1 - \frac{0.05}{2})$, the green lines.
if too many or few observations lie outside these bounds, it would indicate a model problem.

Outliers are observations that are very unlikely according to the distribution, that is, observations that cannot be explained adequately by the model.
The red limits of the can be taken as a guide for classifying outliers. Calculated as the 95\% interval of a standard normal distribution, the Bonferroni corrected for multiple comparisons. 
\begin{equation}
    \text{Outlier limit} = \pm \Phi^{-1}\left(1-\frac{0.05}{2 N}\right)
\end{equation}

where $N$ is the number of observations.

In the example there are no observations outside the limits,

\subsection{\gls{acf} and \gls{pacf}}

The \gls{acf}, top left in \cref{fig:pseudo:correctmodel,fig:pseudo:wrongmodel}, shows the autocorrelation of the pseudo-residuals. The autocorrelation is defined as the correlation between pseudo-residuals at lag k. \cite{shumwayTimeSeriesAnalysis2017}

\begin{equation}
    \text{\gls{acf}}(k) = \text{corr}(z_i,z_{i+k})
\end{equation}

Similarly, the \gls{pacf}, top right plot of \cref{fig:pseudo:correctmodel,fig:pseudo:wrongmodel}, defines the correlation of pseudo-residuals with the linear dependence of previous observations removed\cite{shumwayTimeSeriesAnalysis2017}

\begin{equation}
    \text{\gls{pacf}(k)} = \text{corr}(z_i, z_{i+k} | z_{i+1}, z_{i+2}, \dots ,z_{i+k-1})
\end{equation}

For these plots to make sense, there has to be some structure, spatial, temporal, etc. to the index. In the example there is no autocorrelation, the samples were drawn independently.

\begin{figure}[htb]
    \centering
    \caption[Pseudoresiduals - Correct model]{Diagnostic plots for the Pseudoresiduals of the correctly specified model $y \sim N(0,1^2)$}
    \includegraphics[width=1\linewidth]{Results/Pseudoresiduals/Figures/normal resid.pdf}
    \label{fig:pseudo:correctmodel}
\end{figure}

\begin{figure}[htb]
    \centering
    \caption[Pseudoresiduals - Mispecified model]{Diagnostic plots for the Pseudoresiduals of the mispecified model $y \sim t(2, \mu = 0.2)$}
    \includegraphics[width=1\linewidth]{Results/Pseudoresiduals/Figures/t resid.pdf}
    \label{fig:pseudo:wrongmodel}
\end{figure}