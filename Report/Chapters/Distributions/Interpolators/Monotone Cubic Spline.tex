\subsection{Monotone Cubic Spline}
\label{distributions:spline}

The final interpolation method is the cubic spline interpolation. Compared to quadratic interpolation, the additional polynomial degree ensures that the interpolated function can be both monotone and smooth. 

For the project the interpolation will be done using Scipy's "\href{https://docs.scipy.org/doc/scipy-1.15.3/reference/generated/scipy.interpolate.PchipInterpolator.html#scipy.interpolate.PchipInterpolator}{\color{blue}{Pchipinterpolator}}"\cite{virtanen2020a,InterpolationScipyinterpolateSciPy}

Here, "PCHIP" stands for "Piecewise Cubic Hermite Interpolating Polynomial".

This method constructs a third-order polynomial on each interval

\begin{align}
    f_i(x) &= p_i \phi\left(\frac{t_1}{\Delta Q_i}\right) + p_{i+1} \phi\left(\frac{t_2}{\Delta Q_i}\right) - p_i' \psi\left(\frac{t_1}{\Delta Q_i}\right)\Delta Q_i + p_{i+1}' \psi\left(\frac{t_2}{\Delta Q_i}\right)\Delta Q_i \\
    \phi(x) &=  3x^2 - 2x^2 \\
    \psi(x) &= x^3 - x^2
\end{align}

Where $t_1 = q_{i+1} - x$, $t_2 = x - q_{i}$ and $p'_i$ is the derivative or \gls{pdf} at the point $q_i$.\cite{fritschMonotonePiecewiseCubic1980}

The algorithm estimates these during the fitting process and ensures monotonicity, the details of which are beyond the scope of this project.

Compared with linear methods, the cubic splines interpolation produces a nicer fit, the additional conditions on the derivative improve the interpolation even in regions with sparse data as can be seen in \cref{fig:distributions:examplecdf}.