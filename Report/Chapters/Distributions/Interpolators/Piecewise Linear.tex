\subsection{Piecewise Linear Interpolation}
\label{distributions:pwl}

Piecewise-linear interpolation, as the names suggest, is a function that connects the estimated quantiles in a straight line. 
\begin{equation}
    F_k(x) = ax + b
\end{equation}

To fit a straight line to two points $(q_i,p_o)$ $(q_j,p_j)$ requires solving the equations:

\begin{align}
    F_k(q_i) &= aq_i + b = p_i \\
    F_k(q_{i+1}) &= aq_{i+1} + b = p_{i+1}
\end{align}

Which can quickly be solved giving 
\begin{equation}
\begin{split}
    a &= \frac{p_{i+1}-p_i}{q_{i+1}-q_i} = \frac{\Delta P_1}{\Delta Q_1} \\
    b &= p_i - a q_i = p_i - \frac{\Delta P_1}{\Delta Q_1} q_i
\end{split}
\end{equation}
and with a slight rewriting
\begin{equation}
    f_k(x) = \frac{\Delta P_1}{\Delta Q_1} (x-q_i) + p_i
\end{equation}

The interpolated \gls{cdf} is given by:

\begin{equation}
    F(x) = 
    \begin{cases}
    \hfill 0 \hfill  & x < q_1 \\
    \hfill \frac{\Delta P_1}{\Delta Q_1} (x - q_1) \hfill  & q_1 \leq x < q_2 \\
    \hfill \vdots \hfill & \\
    \hfill \frac{\Delta P_i}{\Delta Q_i} (x - q_i) + p_i\hfill & q_{i} \leq x < q_{i+1}\\
    \hfill \vdots \hfill & \\
    \hfill \frac{\Delta P_{n-1}}{\Delta Q_{n-1}} (x - q_{n-1}) + p_{n-1} & q_{n-1} \leq x < q_{n}\\
    \hfill 1 \hfill& x \geq q_n
    \end{cases}
\end{equation}
The corresponding \gls{pdf} is:
\begin{equation}
    f(x) =
    \begin{cases}
    \hfill 0 \hfill  & x < q_1 \\
    \hfill \frac{\Delta P_1}{\Delta Q_1} \hfill  & q_1 \leq x < q_2 \\
    \hfill \vdots \hfill & \\
    \hfill \frac{\Delta P_i}{\Delta Q_i} \hfill & q_{i} \leq x < q_{i+1}\\
    \hfill \vdots \hfill & \\
    \hfill \frac{\Delta P_{n-1}}{\Delta Q_{n-1}}  & q_{n-1} \leq x < q_{n}\\
    \hfill 0 \hfill& x \geq q_n
    \end{cases}
\end{equation}

Piecewise linear interpolation will generally work quite well in regions where the estimated \gls{cdf} is well described, or where the true \gls{cdf} is roughly linear.

In the tail regions of a distribution, neither is usually the case. This can be seen in \cref{fig:distributions:examplecdf}.

In the well-described middle region, the difference between the interpolation and the true distribution is hard to spot.

In the tail region the difference is evident the linear interpolation in the tail region corresponds to a flat probability in this region, i.e. all values in the tail region is equally likely, see \cref{fig:distributions:examplepdf}.

This leads to the problem described in \cref{distributions:tail}.
