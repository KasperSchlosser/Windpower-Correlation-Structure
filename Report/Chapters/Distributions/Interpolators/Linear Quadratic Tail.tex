\subsection{Piecewise Linear with Quadratic Tail}
\label{distributions:quadratic}

To deal with the problem, \cref{distributions:tail}  it was suggested to increase the order of the interpolation, i.e. piecewise quadratic interpolation.

Monotone quadratic spline have some slightly weird properties. It is not possible to ensure monotonicity, smoothness and that the quantiles match, without insertion of additional interpolation points \cite{schumaker1983a}.

The \gls{pdf} of the interpolation will look rather unusual, essentially consisting of triangular spikes.

The resulting distribution would have the peculiar phenomenon of samples cluster just before or after the observed quantiles. 

The quadratic part was therefore only used in the tails, the rest of the interpolation will be linear.

A quadratic polynomial has 3 parameters:
\begin{equation}
    g(x) = ax^2 +bx + c 
\end{equation}

Fitting this to only two points requires additional constraints. 

The most suitable solution would be to ensure that the derivative at the edges are zero,  $g'(p_0) = 0,\ g'(p_n) = 0$.

To fit the function, we have to solve the equations:
\begin{equation}
\begin{split}
    g(q_1) &= aq_1^2 + bq_i + c = p_1 \\
    g(q_2) &= aq_{2}^2 + bq_{2} + c = p_{2}\\
    g'(q_1) &= 2aq_1 + b = 0
\end{split}
\end{equation}
or the corresponding one for $p_n$.

For the rest of the interpolation points, the method is the same as for the linear interpolator linear, \cref{distributions:pwl}.

The full interpolated \gls{cdf} is given by:
\begin{equation}
    F(x) = 
    \begin{cases}
    \hfill 0 \hfill  & x < q_1 \\
    \hfill \frac{\Delta P_1}{{\Delta Q_1}^2} \left ({x - q_0}\right)^2 \hfill & q_1 \leq x < q_2 \\
    \hfill \vdots \hfill & \\
    \hfill \frac{\Delta P_i}{\Delta Q_i} (x - q_i) + p_i\hfill & q_{i} \leq x < q_{i+1}\\
    \hfill \vdots \hfill  \\
    -\frac{\Delta P_{n-1}}{{\Delta Q_{n-1}}^2} \left (x - q_{n-1}\right)^2 + \frac{\Delta P_{n-1}}{{\Delta Q_{n-1}}} (x - q_{n-1}) + p_{n-1}& q_{n-1} \leq x < q_{n}\\
    \hfill 1 \hfill& x > q_n
    \end{cases}
\end{equation}

and the corresponding \gls{pdf}:
\begin{equation}
    f(x)= 
    \begin{cases}
    \hfill 0 \hfill  & x < q_1 \\
    \hfill 2\frac{\Delta P_1}{{\Delta Q_1}^2} \left ({x - q_0}\right) \hfill & q_1 \leq x < q_2 \\
    \hfill \vdots \hfill & \\
    \hfill \frac{\Delta P_i}{\Delta Q_i}\hfill & q_{i} \leq x < q_{i+1}\\
    \hfill \vdots \hfill  \\
    -2\frac{\Delta P_{n-1}}{{\Delta Q_{n-1}}^2} \left (x - q_{n-1}\right) + \frac{\Delta P_{n-1}}{{\Delta Q_{n-1}}}& q_{n-1} \leq x < q_{n}\\
    \hfill 0 \hfill& x > q_n
    \end{cases}
\end{equation}

The quadratic tail seems to greatly improve the fit in the tail region as can be seen in \cref{fig:distributions:examplecdf}