\section{Interpolators}
\label{distribution:interpolators}

For interpolating a distribution, the initial idea was to keep the method as simple as possible. This resulted in three methods.

\begin{itemize}
    \item Piecwise Linear Method
    \item Piecewise Linear with a quadratic tail
    \item Monotone Cubic spline
\end{itemize}

The initial preference was for a piecewise linear interpolation, but due to some undesirable behaviour, \cref{distributions:tail}, the other methods were included.

To illustrate the interpolation methods, a Dweilbull distribution with shape parameter $c = 2$ is interpolated, and the interpolation points are given in \cref{tab:distributions:exampledata}

\begin{figure}[htb]
    \centering
    \caption[CDF Interpolation]{Interpolation of the \gls{cdf} of a dweibull with shape parameter $c = 2$}
    \includegraphics[width=0.9\linewidth]{Results/Distribution/Figures/Example cdf fit.pdf}
    \label{fig:distributions:examplecdf}
\end{figure}

\begin{figure}[htb]
    \centering
    \caption[PDF Interpolation]{Interpolation of the \gls{pdf} of a dweibull with shape parameter $c = 2$}
    \includegraphics[width=0.9\linewidth]{Results/Distribution/Figures/Example pdf fit.pdf}
    \label{fig:distributions:examplepdf}
\end{figure}

\begin{table}[htb]
\centering
\caption[Example Data Points]{Quantiles an percantages for illustrative example}
\label{tab:distributions:exampledata}
\begin{tabular}{lrrrrrrr}
\toprule
$i$ & 1 & 2 & 3 & 4 & 5 & 6 & 7 \\
\midrule
$q_i$ & -1.000 & -0.300 & -0.200 & -0.100 & 0.000 & 0.300 & 2.000 \\
$p_i$ & 0.184 & 0.457 & 0.480 & 0.495 & 0.500 & 0.543 & 0.991 \\
\bottomrule
\end{tabular}
\end{table}


The interpolated \gls{cdf} can be seen in \cref{fig:distributions:examplecdf}, and the corresponding \gls{pdf} can be seen in \cref{fig:distributions:examplepdf}.

\subsection{Notation}

To improve the readability of the formulas in this section, some specific definitions are used.

The general problem is fitting the function $F$ so that it exactly matches a given set of quantiles with their respective percentages, while ensure it is monotonically nondecreasing $F'(x) \geq 0 $

Denote the vector of quantiles by $Q \in R^n $, the corresponding fraction by $P \in [0;1]^n$, $n$ being the number of pairs.

\begin{equation}
    Q = 
    \begin{bmatrix}
    q_1 \\ q_2 \\ \vdots \\ q_n
    \end{bmatrix},\quad
    P = 
    \begin{bmatrix}
    p_1 \\ p_2 \\ \vdots \\ p_n
    \end{bmatrix},\quad F(q_i) = p_i
\end{equation}

The lagged difference of the quantiles and percentages are denoted by
\begin{equation}
    \Delta Q_i = q_{i+1}-q_i, \quad \Delta P_i = p_{i+1}-p_i
\end{equation}

The values and quantiles are assumed to be ordered and strictly increasing, with the first and last entries corresponding to the 0\% and 100\% quantiles.

\begin{equation}
\begin{split}
    -\infty &< q_1 < q_2 < \dots < q_n < \infty \\
    0 &= p_1 < p_2 < \dots < p_n = 1
\end{split}
\end{equation}

The 0\% and 100\% quantiles will in most cases not correspond the the actual values of the distributions being interpolated.

\subsection{Piecewise Linear Interpolation}
\label{distributions:pwl}

Piecewise-linear interpolation, as the names suggest, is a function that connects the estimated quantiles in a straight line. 
\begin{equation}
    F_k(x) = ax + b
\end{equation}

To fit a straight line to two points $(q_i,p_o)$ $(q_j,p_j)$ requires solving the equations:

\begin{align}
    F_k(q_i) &= aq_i + b = p_i \\
    F_k(q_{i+1}) &= aq_{i+1} + b = p_{i+1}
\end{align}

Which can quickly be solved giving 
\begin{equation}
\begin{split}
    a &= \frac{p_{i+1}-p_i}{q_{i+1}-q_i} = \frac{\Delta P_1}{\Delta Q_1} \\
    b &= p_i - a q_i = p_i - \frac{\Delta P_1}{\Delta Q_1} q_i
\end{split}
\end{equation}
and with a slight rewriting
\begin{equation}
    f_k(x) = \frac{\Delta P_1}{\Delta Q_1} (x-q_i) + p_i
\end{equation}

The interpolated \gls{cdf} is given by:

\begin{equation}
    F(x) = 
    \begin{cases}
    \hfill 0 \hfill  & x < q_1 \\
    \hfill \frac{\Delta P_1}{\Delta Q_1} (x - q_1) \hfill  & q_1 \leq x < q_2 \\
    \hfill \vdots \hfill & \\
    \hfill \frac{\Delta P_i}{\Delta Q_i} (x - q_i) + p_i\hfill & q_{i} \leq x < q_{i+1}\\
    \hfill \vdots \hfill & \\
    \hfill \frac{\Delta P_{n-1}}{\Delta Q_{n-1}} (x - q_{n-1}) + p_{n-1} & q_{n-1} \leq x < q_{n}\\
    \hfill 1 \hfill& x \geq q_n
    \end{cases}
\end{equation}
The corresponding \gls{pdf} is:
\begin{equation}
    f(x) =
    \begin{cases}
    \hfill 0 \hfill  & x < q_1 \\
    \hfill \frac{\Delta P_1}{\Delta Q_1} \hfill  & q_1 \leq x < q_2 \\
    \hfill \vdots \hfill & \\
    \hfill \frac{\Delta P_i}{\Delta Q_i} \hfill & q_{i} \leq x < q_{i+1}\\
    \hfill \vdots \hfill & \\
    \hfill \frac{\Delta P_{n-1}}{\Delta Q_{n-1}}  & q_{n-1} \leq x < q_{n}\\
    \hfill 0 \hfill& x \geq q_n
    \end{cases}
\end{equation}

Piecewise linear interpolation will generally work quite well in regions where the estimated \gls{cdf} is well described, or where the true \gls{cdf} is roughly linear.

In the tail regions of a distribution, neither is usually the case. This can be seen in \cref{fig:distributions:examplecdf}.

In the well-described middle region, the difference between the interpolation and the true distribution is hard to spot.

In the tail region the difference is evident the linear interpolation in the tail region corresponds to a flat probability in this region, i.e. all values in the tail region is equally likely, see \cref{fig:distributions:examplepdf}.

This leads to the problem described in \cref{distributions:tail}.

\subsection{Piecewise Linear with Quadratic Tail}
\label{distributions:quadratic}

To deal with the problem, \cref{distributions:tail}  it was suggested to increase the order of the interpolation, i.e. piecewise quadratic interpolation.

Monotone quadratic spline have some slightly weird properties. It is not possible to ensure monotonicity, smoothness and that the quantiles match, without insertion of additional interpolation points \cite{schumaker1983a}.

The \gls{pdf} of the interpolation will look rather unusual, essentially consisting of triangular spikes.

The resulting distribution would have the peculiar phenomenon of samples cluster just before or after the observed quantiles. 

The quadratic part was therefore only used in the tails, the rest of the interpolation will be linear.

A quadratic polynomial has 3 parameters:
\begin{equation}
    g(x) = ax^2 +bx + c 
\end{equation}

Fitting this to only two points requires additional constraints. 

The most suitable solution would be to ensure that the derivative at the edges are zero,  $g'(p_0) = 0,\ g'(p_n) = 0$.

To fit the function, we have to solve the equations:
\begin{equation}
\begin{split}
    g(q_1) &= aq_1^2 + bq_i + c = p_1 \\
    g(q_2) &= aq_{2}^2 + bq_{2} + c = p_{2}\\
    g'(q_1) &= 2aq_1 + b = 0
\end{split}
\end{equation}
or the corresponding one for $p_n$.

For the rest of the interpolation points, the method is the same as for the linear interpolator linear, \cref{distributions:pwl}.

The full interpolated \gls{cdf} is given by:
\begin{equation}
    F(x) = 
    \begin{cases}
    \hfill 0 \hfill  & x < q_1 \\
    \hfill \frac{\Delta P_1}{{\Delta Q_1}^2} \left ({x - q_0}\right)^2 \hfill & q_1 \leq x < q_2 \\
    \hfill \vdots \hfill & \\
    \hfill \frac{\Delta P_i}{\Delta Q_i} (x - q_i) + p_i\hfill & q_{i} \leq x < q_{i+1}\\
    \hfill \vdots \hfill  \\
    -\frac{\Delta P_{n-1}}{{\Delta Q_{n-1}}^2} \left (x - q_{n-1}\right)^2 + \frac{\Delta P_{n-1}}{{\Delta Q_{n-1}}} (x - q_{n-1}) + p_{n-1}& q_{n-1} \leq x < q_{n}\\
    \hfill 1 \hfill& x > q_n
    \end{cases}
\end{equation}

and the corresponding \gls{pdf}:
\begin{equation}
    f(x)= 
    \begin{cases}
    \hfill 0 \hfill  & x < q_1 \\
    \hfill 2\frac{\Delta P_1}{{\Delta Q_1}^2} \left ({x - q_0}\right) \hfill & q_1 \leq x < q_2 \\
    \hfill \vdots \hfill & \\
    \hfill \frac{\Delta P_i}{\Delta Q_i}\hfill & q_{i} \leq x < q_{i+1}\\
    \hfill \vdots \hfill  \\
    -2\frac{\Delta P_{n-1}}{{\Delta Q_{n-1}}^2} \left (x - q_{n-1}\right) + \frac{\Delta P_{n-1}}{{\Delta Q_{n-1}}}& q_{n-1} \leq x < q_{n}\\
    \hfill 0 \hfill& x > q_n
    \end{cases}
\end{equation}

The quadratic tail seems to greatly improve the fit in the tail region as can be seen in \cref{fig:distributions:examplecdf}
\subsection{Monotone Cubic Spline}
\label{distributions:spline}

The final interpolation method is the cubic spline interpolation. Compared to quadratic interpolation, the additional polynomial degree ensures that the interpolated function can be both monotone and smooth. 

For the project the interpolation will be done using Scipy's "\href{https://docs.scipy.org/doc/scipy-1.15.3/reference/generated/scipy.interpolate.PchipInterpolator.html#scipy.interpolate.PchipInterpolator}{\color{blue}{Pchipinterpolator}}"\cite{virtanen2020a,InterpolationScipyinterpolateSciPy}

Here, "PCHIP" stands for "Piecewise Cubic Hermite Interpolating Polynomial".

This method constructs a third-order polynomial on each interval

\begin{align}
    f_i(x) &= p_i \phi\left(\frac{t_1}{\Delta Q_i}\right) + p_{i+1} \phi\left(\frac{t_2}{\Delta Q_i}\right) - p_i' \psi\left(\frac{t_1}{\Delta Q_i}\right)\Delta Q_i + p_{i+1}' \psi\left(\frac{t_2}{\Delta Q_i}\right)\Delta Q_i \\
    \phi(x) &=  3x^2 - 2x^2 \\
    \psi(x) &= x^3 - x^2
\end{align}

Where $t_1 = q_{i+1} - x$, $t_2 = x - q_{i}$ and $p'_i$ is the derivative or \gls{pdf} at the point $q_i$.\cite{fritschMonotonePiecewiseCubic1980}

The algorithm estimates these during the fitting process and ensures monotonicity, the details of which are beyond the scope of this project.

Compared with linear methods, the cubic splines interpolation produces a nicer fit, the additional conditions on the derivative improve the interpolation even in regions with sparse data as can be seen in \cref{fig:distributions:examplecdf}.
\subsection{Other Interpolators}
\label{distributions:other}

\subsubsection{Piecewise Constant Interpolation}

The simplest interpolation method would be a piecewise-constant interpolation.

\begin{equation}
    F(x) =
    \begin{cases}
        0 & x < q_1 \\
        p_1 & q_1 \leq x < q_2 \\
        \hfill\vdots \hfill\\
        p_{i+1} - p_{i-1}& q_{i-1} \leq x < q_i \\
        \hfill\vdots \hfill\\
        p_{n} & q_{n-1} \leq q_{n} \\
        1 & q_{n} < x
    \end{cases}
\end{equation}

This method essentially discretises the outcome space into a discrete random variable with outcomes $n-1$.

\begin{equation}
    F(x = k) =
    \begin{cases}
        p_2 & k = q_1 \\
        \vdots \\
        p_{i+1} - p_i & k = q_i\\
        \vdots \\
        1 - p_{n} & k = q_{n} \\
    \end{cases}
\end{equation}
Or variations of these.
If the outcome space is easy to categories, e.g. if we are only interested in being above/below the median. This formulation could be very useful due to the simplicity and easy intepretation.

The downside is that the resulting distribution is discrete. For the investigation of autocorrelation, a discrete distribution would not be sufficient and this method is not used further.

\subsubsection{Monotone Quintic Spline Interpolation}

A problem with the 3 methods discussed in \cref{distributions:pwl,distributions:quadratic,distributions:spline} is in the estimated\gls{pdf} when using these methods.

The interpolated \gls{pdf} can have a strange look, the spline \gls{pdf} can be very "bubbly", and worse are not guaranteed to be continuous or smooth\footnote{Defined as having continuous first derivative}, both properties would be expected for common distributions.

With quintic splines, it is possible to construct an interpolation that is monotone, continuous, and with a smooth first derivative, giving the properties expected\cite{luxAlgorithm1031MQSI2023}. 

However, the literature on this topic is very sparse, as also noted by \textcite{luxAlgorithm1031MQSI2023}. Implementations are even sparser, and the few implementations that exist are implemented for Fortran \cite{luxAlgorithm1031MQSI2023}.

Although the properties of higher-order splines would be appealing to use, due to the lack of implementation, they are not used for this project.
