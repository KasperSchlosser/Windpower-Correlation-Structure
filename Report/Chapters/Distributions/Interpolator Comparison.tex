\section{Interpolator Comparison}
\label{distributions:comparison}

To compare the three interpolation methods, 4 test distributions were selected.

\begin{itemize}
    \item Uniform distributions $[-6;6]$
    \item Standard normal distribution $N(0,1^2)$
    \item Student's t distribution $t(2)$
    \item Beta distribution $Beta(5,3)$ and adjusted to $[-6,6]$
\end{itemize}
The scaling of the distribution was chosen mainly for visualisation purposes.
A plot of the \gls{cdf}s can be seen in \cref{fig:distributions:comparison-cdf}, and the corresponding \gls{pdf} in \cref{fig:distribution:fittedpdf}

\begin{figure}[htb]
    \centering
    \caption[CDF of the Test Distributions]{The \gls{cdf} of the four test distributions: a uniform distribution $U(-6,6)$; a standard normal distribution $N(0,1^2)$; a Student's t distribution $t(2)$; and a beta distribution $Beta(5,3)$, adjusted to the interval (-6,6).}
    \includegraphics[width=1\linewidth]{Results/Distribution/Figures/Comparison cdf.pdf}
    \label{fig:distributions:comparison-cdf}
\end{figure}

\begin{figure}[htb]
    \centering
    \caption[PDF of the Test Distributions]{The \gls{pdf} of the four test distributions: a uniform distribution $U(-6,6)$; a standard normal distribution $N(0,1^2)$; a Student's t distribution $t(2)$; and a beta distribution $Beta(5,3)$, adjusted to the interval (-6,6).}
    \includegraphics[width=1\linewidth]{Results/Distribution/Figures/Comparison pdf.pdf}
    \label{fig:distributions:comparison-pdf}
\end{figure}

The distributions were chosen to have a mix of heavy- and light-tailed, the t- and normal-distributions, respectively, a skewed distribution, the beta distribution, and the uniform distribution as a very simple test distribution.

For interpolation, the 25\%, 50\%, and 75\% quantile of the distributions were chosen as interpolation points. To limit the interpolations, points -6 and 6 were chosen as the 0\% and 100\% quantiles, respectively, for all distributions. 

The actual values can be found in \cref{tab:distributions:testinterpolation}

\begin{table}[htb]
\centering
\caption[Test interpolation points]{The interpoilation points used for testing the interpolation methods}
\label{tab:distributions:testinterpolation}
\begin{tabular}{lrrrrr}
\toprule
 & $F^{-1}(0.00)$ & $F^{-1}(0.25)$ & $F^{-1}(0.50)$ & $F^{-1}(0.75)$ & $F^{-1}(1.00)$ \\
\midrule
Uniform & -6.00 & -3.00 & 0.00 & 3.00 & 6.00 \\
Normal & -6.00 & -0.67 & 0.00 & 0.67 & 6.00 \\
Student's t & -6.00 & -0.82 & 0.00 & 0.82 & 6.00 \\
Beta & -6.00 & 0.17 & 1.63 & 2.96 & 6.00 \\
\bottomrule
\end{tabular}
\end{table}


\newpage
\subsection{Comparison}

After interpolation, the Wasserstein distance with $p = 1$ and $p = 2$ and \gls{kl} between the interpolated distribution and the true distributions were calculated.

The results can be seen in \cref{tab:distributions:testres}, while the actual interpolations are visualised in \cref{appendix:distributions}.

\begin{table}[htb]
\centering
\caption[Interpolation Scores]{The Wasserstein distance (with \( p = 1 \) and \( p = 2 \)) and \gls{kl} for the three interpolation methods.}
\label{tab:distributions:testres}
\begin{tabular}{lllll}
\toprule
 &  & Linear & Linear - tail & Spline \\
\midrule
\multirow[r]{3}{*}{Uniform} & $W_1$ & \bfseries 0.00 & 0.25 & \bfseries 0.00 \\
 & $W_2$ & \bfseries 0.00 & 0.39 & \bfseries 0.00 \\
 & $D_{KL}$ & \bfseries 0.00 & 0.10 & \bfseries 0.00 \\
\cline{1-5}
\multirow[r]{3}{*}{Normal} & $W_1$ & 1.04 & 0.60 & \bfseries 0.55 \\
 & $W_2$ & 1.65 & 1.00 & \bfseries 0.93 \\
 & $D_{KL}$ & 2.31 & 0.92 & \bfseries 0.82 \\
\cline{1-5}
\multirow[r]{3}{*}{Student's t} & $W_1$ & 0.83 & 0.45 & \bfseries 0.43 \\
 & $W_2$ & 2.77 & 2.62 & \bfseries 2.62 \\
 & $D_{KL}$ & 0.36 & 0.13 & \bfseries 0.12 \\
\cline{1-5}
\multirow[r]{3}{*}{Beta} & $W_1$ & 0.62 & \bfseries 0.23 & 0.30 \\
 & $W_2$ & 1.07 & \bfseries 0.47 & 0.53 \\
 & $D_{KL}$ & 0.64 & \bfseries 0.10 & 0.14 \\
\cline{1-5}
\bottomrule
\end{tabular}
\end{table}


Unsurprisingly, the piecewise linear interpolation perfectly matched the uniform distribution. Slightly more interesting is that the spline interpolation also matched the uniform distribution perfectly.

We can see that except for the beta distribution the spline models is the best model. 

The difference in the beta distribution seems to come from the fact that the right tail of the distribution almost follows a quadratic curve, see \cref{fig:distributions:beta-ex-pdf}, and the quadratic tail model scores better.

The linear model with a quadratic tail term performs quite well. The result would indicate that when the main challenge is in estimate tail distribution, there is little difference between the two methods.

The spline interpolations however has slightly better scores, and the continuous \gls{pdf} it produces is a desireable advantage. The spline model is therefor used in this project

\begin{figure}[hbt]
    \centering
    \caption[PDF of the Beta Distribution]{The \gls{pdf} of the $Beta(5,3)$ along with the interpolated distributions.}
    \includegraphics[width=1\linewidth]{Results/Distribution/Figures/Comparison pdf.pdf}
    \label{fig:distributions:beta-ex-pdf}
\end{figure}