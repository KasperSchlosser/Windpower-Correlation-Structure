\section{Tail Problem}
\label{distributions:tail}

A problem discovered during the course of the project was dubbed the "Tail problem". The prediction interval for some of the models became very "spiky" in a way that did not seem plausible.

An simulated example can be seen in \cref{fig:distributions:tail}, where a time-varying process is forecasted.

\begin{figure}[htb]
    \centering
    \caption[Tail Problem]{Example of the tail problem. Forecast distribution is interpolated with the bounds -10, and 10, interval is 90\%}
    \includegraphics[width=1\linewidth]{Results/Distribution/Figures/Tail Problem.pdf}
    \label{fig:distributions:tail}
\end{figure}

The problem is caused by the interpolation methods. The methods need some bounds, and these bounds might be quite far from the true value.

Take a linear interpolation on the normal distribution, with the interpolation points $q_1, = F^{-1}(0.975) \approx 2 $ and and upper bounds set to $q_2 = F^{-1}(1) := 10$. The interpolated 99.9\% quantile would be 
\begin{equation}
    \hat{F}^{-1}(0.999) = \frac{q_2-q_1}{p_2-p_1}(0.999 - p_1) + p_2 \approx \frac{6-2}{1-0.975}(0.999 - 0.975) + 2 = 5.84
\end{equation}

However, the true quantile is $F^{-1}(0.999) \approx 3 $.

The problem is most pronounced with linear interpolation; the other interpolation methods were investigated to help alleviate the problem. As can be seen in \cref{fig:distributions:tail} the higher order methods attenuate the problem but do not completely eliminate it.

As this problem only affects the tail of the distribution, the actual impact of the problem should be minor and mostly be a visual nuisance.