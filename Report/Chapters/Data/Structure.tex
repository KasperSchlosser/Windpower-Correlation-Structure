\chapter{Data}
\label{data}
\addcontentsline{lof}{chapter}{Data}
\addcontentsline{lot}{chapter}{Data}

This thesis builds on the research of \textcite{jorgensenSequentialMethodsError2025}, using the same data set.

The entire data set spans an approximately three-year period from January 2022 to mid October 2024 with the specific dates given in \cref{tab:datadate}

\begin{table}[htb]
    \centering
    \caption[Data Period]{Data Period}
    \label{tab:datadate}
    \begin{tabular}{@{}ll@{}}
    \toprule
    Start date & End date \\ \midrule
    2022-01-01 00:00:00 & 2024-10-12 23:00:00 \\ \bottomrule
    \end{tabular}
\end{table}

The data set consists of hourly observations of wind power production for each of the four zones, detailed in \cref{data:production}.

In addition to observation data, the dataset also includes hourly production forecasts, again in each of the four zones. Forecast data is described in \cref{data:ensemble}.

All data, both production and forecast, are given in MWh.

In the thesis by \textcite{jorgensenRealTimeForecastingRenewable2024} a larger effort was made to deal with two things in the dataset. 

Firstly, due to agreements between Denmark and other countries, Danish producers are sometime paid to shutdown production, known as curtailment.

As changes the agreements this is curtailment is expected to be less of a problem in the future. Periods with curtailment was therefore removed as these periods would not be comparable with future behaviour.

Secondly, The forecast for DK2-offshore had some obvious failure modes, described in \cref{data:forecastfailure}, These periods was also removed.

For this project it was decided not to remove these periods, Firstly, Removing the data would mean there would be jumps in observations which would interfere with learning the correlations structure.

Secondly it was decided that removing the failures in DK2-offshore would artificially improve the performance of the base forecast. The forecasts should be evaluated on the actual performance, not the ideal performance.


\subsection*{Data Alignment}

One small detail is the alignment of the data.
The forecast data were provided with time zone information, while the production data were not. To align, the time-zone was simply dropped from the ensemble data, which is what seemed to have been done in \textcite{jorgensenSequentialMethodsError2025}

\newpage
\section{Wind-power-Production}
\label{data:production}

\subsection{Energi Data Service}

Power production data was acquired through  \href{https://www.energidataservice.dk/}{\color{blue}{Energi Data Service}}, an open data platform provided by Energinet\cite{energinetEnergiDataService}.

\begin{displayquote}[\cite{energinetEnergiDataService}]
Energi Data Service is a free and open data portal. Here, anyone can get data about the Danish energy system, such as CO2 emissions and consumption and production data.
\end{displayquote}

Energinet is an independent public enterprise that owns and maintains the national Danish electricity and gas grids \cite{Energinet2025}

\subsection{Project Data}

For this thesis, data were acquired on request from the author of \textcite{jorgensenRealTimeForecastingRenewable2024}. Originally, these data were acquired from the "\href{https://www.energidataservice.dk/tso-electricity/ElectricityProdex5MinRealtime}{\color{blue}{Electricity Production and Exchange 5 min Realtime}\cite{energinetHome}}"-data set provided by Energi Data Service.

Original data is power production given in 5 minute increments. Hourly data was obtained as the mean production over the hour.

As noted on the website the 5-minute observation come from upscaled real-time measurements, and data errors may occur\cite{energinetHome}

Definite data errors are negative values of observed power production, in addition some NaN values has been introduced to dataset, possibly caused by the handling of the time zone information.

The data was split according to bidding zone, and wether production happend onshore or offshore giving four data sets

\begin{itemize}
    \item DK1-onshore
    \item DK1-offshore
    \item DK2-onshore
    \item DK2-offshore
\end{itemize}

Each zone has different production capacity, DK1-onshore being by far the largest zone, with same maximal capacity of the other three zones combined.

A small example of the data can be seen in \cref{fig:data:production-small}. 
Here it can be noticed there is a correlation between zones, with the production in one zone roughly following the production in other zones. The fulld data can be found the the appendix \cref{fig:data:production-full}.

\begin{figure}[htb]
    \centering
    \caption[Production data]{Production data from  17. Oct 2023. to 1. Nov. 2023}
    \includegraphics[width=1\linewidth]{Results/Data/Figures/Comparison small.pdf}
    \label{fig:data:production-small}
\end{figure}


\subsubsection{Data cleaning}

To clean the data all NaN values are simply dropped.
As some of the models and methods used in this have problems handling observations very close to zero, a small value $0.01$ was chosen as effectively zero.

A summary of raw power production data can be seen in \cref{tab:data:observation}

\begin{table}[htb]
\centering
\caption[Summary table for Production data]{Summary table for the observed production data. Problematic values are values $0 \leq x < 0.01$}
% \begin{tabular}{lrrrrrr}
\begin{tabular}{lcccccc}
\toprule
 & Count & Minimum & Maximum & Negative & Problematic & NaN \\
\midrule
DK1-offshore & 24384 & -8.93 & 1212.79 & 348 & 2 & 0 \\
DK1-onshore & 24381 & 4.40 & 3562.69 & 0 & 0 & 3 \\
DK2-offshore & 24381 & -6.34 & 982.52 & 742 & 5 & 3 \\
DK2-onshore & 24381 & 5.20 & 634.85 & 0 & 0 & 3 \\
\bottomrule
\end{tabular}
\label{tab:data:observation}
\end{table}


To clean the data  The observation data all NaN values were simply dropped. Then all values smaller than $0.01$ was set to $0.01$.

\section{Forecast Data}
\label{data:ensemble}

The previous work by \textcite{jorgensenSequentialMethodsError2025} improved on a base forecast, using purely mathematical mathematical methods. This thesis uses the same base forecast for comparison.

The forecast is based on the 15-day ensemble weather forecast by \gls{ecmwf} This forecast consist a \gls{hpe}, the scenario deemed most likely to happen, along with alternative scenarios made by pertubing the inital condition of the \gls{hpe}.

In the base forecast each \gls{ecmwf} ensemble is converted into a forecast power production. The exact method is proprietary, so the details of calculation are not provided, likewise the actual forecasts cannot be made available. 

\subsection{ECMWF}

The \gls{ecmwf} is a European organization that conducts research and offers weather forecasts, as detailed on their website. 

\begin{displayquote}[\textcite{ECMWF2025}]
We are both a research institute and a 24/7 operational service, producing global numerical weather predictions and other data for our Member and Co-operating States and the broader community.
\end{displayquote}

The 15-day ensemble forecast has a one-hour resolution up to 90 hours ahead. and a three-hour resolution from hour 90 to 144\cite{ECMWF2025}.

Forecasts are made four times a day at 00:00, 06:00, 12:00, and 18:00, with the complete forecast for a given day available 12 hours before the day\cite{ECMWF2025}.

\subsection{Ensemble Forecast}

For the actual forecast data used, the data-sets, split equivalently to the observation data, was aquired from \textcite{jorgensenSequentialMethodsError2025}. No immediate problems seemed to exist in the data, no NaN values.

While negative forecast are probably not ideal, given negative power is no possible, They do not represent any problem. a summary of the forecast data can be seen in \cref{tab:data:ensemble}

\begin{table}[h]
\centering
\caption[Summary table for Ensemble data]{Summary table for Ensemble data}
\begin{tabular}{lrrrr}
\toprule
 & Ensembles & Count & Minimum & Maximum \\
\midrule
DK1-offshore & 52 & 24833 & 0.00 & 1080.98  \\
DK1-onshore & 52 & 24833 & 4.95 & 3324.15 \\
DK2-offshore & 52 & 24833 & 0.00 & 1034.17 \\
DK2-onshore & 52 & 24833 & 0.00 & 854.38 \\
\bottomrule
\end{tabular}
\label{tab:data:ensemble}
\end{table}


An example of the ensemble forecasts can be see in \cref{fig:data:dk1onshore-ex,fig:data:dk1offshore-ex,fig:data:dk2onshore-ex,fig:data:dk2offshore-ex}. while the complete dataset can be seen in the appendix \cref{appendix:data}

In general the offshore forecast tend to exhibit higher volatility.

\subsection{Failure Modes}
\label{data:forecastfailure}

While no immediate problems was evident, One problem exists

For DK2-offshore the forecasts has a failure mode where all ensembled collapse around a specific value, $\sim400$. Which can be seen in \cref{fig:data:dk2offshore-ex} around the 2023-10-20.

\begin{figure}[htb]
    \centering
    \caption[DK1-onshore]{Example of thethe ensemble forecast in DK1-onshore}
    \label{fig:data:dk1onshore-ex}
    \includegraphics[width=0.9\linewidth]{Results/Data/Figures/DK1-onshore small.pdf}
\end{figure}

\begin{figure}[htb]
    \centering
    \caption[DK1-offshore]{Example of the the ensemble forecast in DK1-onshore}
    \label{fig:data:dk1offshore-ex}
    \includegraphics[width=0.9\linewidth]{Results/Data/Figures/DK1-offshore small.pdf}
\end{figure}

\begin{figure}[htb]
    \centering
    \caption[DK2-onshore]{Example of the the ensemble forecast in DK2-onshore}
    \label{fig:data:dk2onshore-ex}
    \includegraphics[width=0.9\linewidth]{Results/Data/Figures/DK2-onshore small.pdf}
\end{figure}

\begin{figure}[htb]
    \centering
    \caption[DK2-offshore]{Example of the ensemble forecast in DK2-offshore}
    \label{fig:data:dk2offshore-ex}
    \includegraphics[width=0.9\linewidth]{Results/Data/Figures/DK2-offshore small.pdf}
\end{figure}
\newpage
\section{Data-Split}
\label{data:datasplit}

\subsection{Training and Evaluation}

To avoid biases and detect overfitting, it is essential to train and evaluate models on separate datasets. Ensuring that no direct information is shared between the training and evaluation phases, other than the underlying system of interest, is crucial\cite{hastieElementsStatisticalLearning2009}.

This separation helps prevent models from overfitting, similar to memorising an answer sheet for a test, which can lead to artificially inflated performance metrics. \cite{hastieElementsStatisticalLearning2009}.

A simple and common way of handling this is the k-fold cross-validation. Where the data are (randomly) divided into $k$ blocks, the models are trained in $k-1$ blocks and evaluated on the last block, one fold. Then repeated for each block\cite{hastieElementsStatisticalLearning2009}.

Because of the high temporal correlation that exists in the data, this approach does not work. Data leaks can occur as future data correlate with current.\cite{robertsCrossvalidationStrategiesData2017}.

Ideally, more advanced strategies are used as described in \textcite{robertsCrossvalidationStrategiesData2017}, however, due to time and computational constraints, a simple train-test split was chosen.

\subsection{Train-Test Split}

As wind power is heavily dependent on weather, power production tends to show a seasonal effect over the year. To account for this seasonal effect, the data split was chosen as 64\%-36\% (train-test). As this split gave roughly one year of evaluation data. 

The details of the split can be seen in \cref{tab:data:split} and is illustrated in \cref{fig:data:datasplit}

\begin{figure}[htb]
    \centering
    \caption[Train-Test Split]{Train-test split of data. Visualized
    for DK1-onshore}
    \includegraphics[width=1\linewidth]{Results/Data/Figures/Datasplit.pdf}
    \label{fig:data:datasplit}
\end{figure}

\section{Data-Split}
\label{data:datasplit}

\subsection{Training and Evaluation}

To avoid biases and detect overfitting, it is essential to train and evaluate models on separate datasets. Ensuring that no direct information is shared between the training and evaluation phases, other than the underlying system of interest, is crucial\cite{hastieElementsStatisticalLearning2009}.

This separation helps prevent models from overfitting, similar to memorising an answer sheet for a test, which can lead to artificially inflated performance metrics. \cite{hastieElementsStatisticalLearning2009}.

A simple and common way of handling this is the k-fold cross-validation. Where the data are (randomly) divided into $k$ blocks, the models are trained in $k-1$ blocks and evaluated on the last block, one fold. Then repeated for each block\cite{hastieElementsStatisticalLearning2009}.

Because of the high temporal correlation that exists in the data, this approach does not work. Data leaks can occur as future data correlate with current.\cite{robertsCrossvalidationStrategiesData2017}.

Ideally, more advanced strategies are used as described in \textcite{robertsCrossvalidationStrategiesData2017}, however, due to time and computational constraints, a simple train-test split was chosen.

\subsection{Train-Test Split}

As wind power is heavily dependent on weather, power production tends to show a seasonal effect over the year. To account for this seasonal effect, the data split was chosen as 64\%-36\% (train-test). As this split gave roughly one year of evaluation data. 

The details of the split can be seen in \cref{tab:data:split} and is illustrated in \cref{fig:data:datasplit}

\begin{figure}[htb]
    \centering
    \caption[Train-Test Split]{Train-test split of data. Visualized
    for DK1-onshore}
    \includegraphics[width=1\linewidth]{Results/Data/Figures/Datasplit.pdf}
    \label{fig:data:datasplit}
\end{figure}

\section{Data-Split}
\label{data:datasplit}

\subsection{Training and Evaluation}

To avoid biases and detect overfitting, it is essential to train and evaluate models on separate datasets. Ensuring that no direct information is shared between the training and evaluation phases, other than the underlying system of interest, is crucial\cite{hastieElementsStatisticalLearning2009}.

This separation helps prevent models from overfitting, similar to memorising an answer sheet for a test, which can lead to artificially inflated performance metrics. \cite{hastieElementsStatisticalLearning2009}.

A simple and common way of handling this is the k-fold cross-validation. Where the data are (randomly) divided into $k$ blocks, the models are trained in $k-1$ blocks and evaluated on the last block, one fold. Then repeated for each block\cite{hastieElementsStatisticalLearning2009}.

Because of the high temporal correlation that exists in the data, this approach does not work. Data leaks can occur as future data correlate with current.\cite{robertsCrossvalidationStrategiesData2017}.

Ideally, more advanced strategies are used as described in \textcite{robertsCrossvalidationStrategiesData2017}, however, due to time and computational constraints, a simple train-test split was chosen.

\subsection{Train-Test Split}

As wind power is heavily dependent on weather, power production tends to show a seasonal effect over the year. To account for this seasonal effect, the data split was chosen as 64\%-36\% (train-test). As this split gave roughly one year of evaluation data. 

The details of the split can be seen in \cref{tab:data:split} and is illustrated in \cref{fig:data:datasplit}

\begin{figure}[htb]
    \centering
    \caption[Train-Test Split]{Train-test split of data. Visualized
    for DK1-onshore}
    \includegraphics[width=1\linewidth]{Results/Data/Figures/Datasplit.pdf}
    \label{fig:data:datasplit}
\end{figure}

\input{Results/Data/Tables/Datasplit}

All models used in this project, with one exception, is trained exclusively on the train-set. No data leakage should occur from the test set.

The one exception is the \gls{taqr} model, explained in \cref{basis:taqr}. As an adaptive model, the model could only give forecasts in a streaming manner with continuously updating estimates.









All models used in this project, with one exception, is trained exclusively on the train-set. No data leakage should occur from the test set.

The one exception is the \gls{taqr} model, explained in \cref{basis:taqr}. As an adaptive model, the model could only give forecasts in a streaming manner with continuously updating estimates.









All models used in this project, with one exception, is trained exclusively on the train-set. No data leakage should occur from the test set.

The one exception is the \gls{taqr} model, explained in \cref{basis:taqr}. As an adaptive model, the model could only give forecasts in a streaming manner with continuously updating estimates.








\section{Data Normalization}

Although not strictly necessary, normalising or scaling data have been noted to improve model convergence and in some cases performance. 

The ensemble and input data were normalised differently and independently of each other.

\subsection{Ensemble Normalisation}

The ensemble data was normalized two different ways. As a comparison is made with the original \gls{nabqr} model, described in \cref{basis:nabqr}. For the training of this model, the ensembles where scaled to the range $[0;1]$ by the formula:

\begin{equation}
    \tilde{x}_{t} = \frac{x_{t} - \min_i x_{i} }{\max_i x_{i}-\min_i x_{i} }
\end{equation}

Where $\tilde{x}_{t}$ if the scaled data at time t and $x_t$ is the data a time t

For other models, the ensemble data was scaled to the range $[-1;1]$ by the almost equivalent formula:

\begin{equation}
    \tilde{x}_{t} = 2 \frac{x_{t} - \min_i x_{i} }{\max_i x_{i}-\min_i x_{i} } - 1
\end{equation}

Each zone is scaled independently of each other, and the respective minimums and maximums were taken only over the train-set.

\subsection{Observation Data}

For the observation data. The scaled data would only be used intermittently, only used for calculating loss during model training to improve convergence. Less care was therefore put into choosing the scaling.

The observation data was transformed using the formula:

\begin{equation}
    \tilde{y}_{t} = \frac{y_{t}}{ c_{\text{zone}}} 
\end{equation}

Where $y_t$ is the actual observation, $\tilde{y}_{t}$ is the scaled observation, and $ c_{\text{zone}}$ is a constant for each zone: $c_{DK1-offshore} = 1250$, $c_{DK1-onshore} = 3600$, $c_{DK2-offshore} = 100$, $c_{DK2-onshore} = 650$







