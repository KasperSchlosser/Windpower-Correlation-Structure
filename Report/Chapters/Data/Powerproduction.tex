\section{Wind-power-Production}
\label{data:production}

\subsection{Energi Data Service}

Power production data was acquired through  \href{https://www.energidataservice.dk/}{\color{blue}{Energi Data Service}}, an open data platform provided by Energinet\cite{energinetEnergiDataService}.

\begin{displayquote}[\cite{energinetEnergiDataService}]
Energi Data Service is a free and open data portal. Here, anyone can get data about the Danish energy system, such as CO2 emissions and consumption and production data.
\end{displayquote}

Energinet is an independent public enterprise that owns and maintains the national Danish electricity and gas grids \cite{Energinet2025}

\subsection{Project Data}

For this thesis, data were acquired on request from the author of \textcite{jorgensenRealTimeForecastingRenewable2024}. Originally, these data were acquired from the "\href{https://www.energidataservice.dk/tso-electricity/ElectricityProdex5MinRealtime}{\color{blue}{Electricity Production and Exchange 5 min Realtime}\cite{energinetHome}}"-data set provided by Energi Data Service.

Original data is power production given in 5 minute increments. Hourly data was obtained as the mean production over the hour.

As noted on the website the 5-minute observation come from upscaled real-time measurements, and data errors may occur\cite{energinetHome}

Definite data errors are negative values of observed power production, in addition some NaN values has been introduced to dataset, possibly caused by the handling of the time zone information.

The data was split according to bidding zone, and wether production happend onshore or offshore giving four data sets

\begin{itemize}
    \item DK1-onshore
    \item DK1-offshore
    \item DK2-onshore
    \item DK2-offshore
\end{itemize}

Each zone has different production capacity, DK1-onshore being by far the largest zone, with same maximal capacity of the other three zones combined.

A small example of the data can be seen in \cref{fig:data:production-small}. 
Here it can be noticed there is a correlation between zones, with the production in one zone roughly following the production in other zones. The fulld data can be found the the appendix \cref{fig:data:production-full}.

\begin{figure}[htb]
    \centering
    \caption[Production data]{Production data from  17. Oct 2023. to 1. Nov. 2023}
    \includegraphics[width=1\linewidth]{Results/Data/Figures/Comparison small.pdf}
    \label{fig:data:production-small}
\end{figure}


\subsubsection{Data cleaning}

To clean the data all NaN values are simply dropped.
As some of the models and methods used in this have problems handling observations very close to zero, a small value $0.01$ was chosen as effectively zero.

A summary of raw power production data can be seen in \cref{tab:data:observation}

\begin{table}[htb]
\centering
\caption[Summary table for Production data]{Summary table for the observed production data. Problematic values are values $0 \leq x < 0.01$}
% \begin{tabular}{lrrrrrr}
\begin{tabular}{lcccccc}
\toprule
 & Count & Minimum & Maximum & Negative & Problematic & NaN \\
\midrule
DK1-offshore & 24384 & -8.93 & 1212.79 & 348 & 2 & 0 \\
DK1-onshore & 24381 & 4.40 & 3562.69 & 0 & 0 & 3 \\
DK2-offshore & 24381 & -6.34 & 982.52 & 742 & 5 & 3 \\
DK2-onshore & 24381 & 5.20 & 634.85 & 0 & 0 & 3 \\
\bottomrule
\end{tabular}
\label{tab:data:observation}
\end{table}


To clean the data  The observation data all NaN values were simply dropped. Then all values smaller than $0.01$ was set to $0.01$.
