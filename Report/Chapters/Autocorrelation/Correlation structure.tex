\section{Pseudoresidual Correlation Structure}

For the pipeline correlation model, a \gls{sarma} model was used. To select the most suitable order of the model, a search gridsearch was performed over all possible combinations of models 
\begin{gather*}
    SARMA( (p,0,q), (P,0,Q,24)) \\
    p,q \in \{0,1,2,3\} \\
    P, Q\in \{0,1,2\}
\end{gather*}

Each model Was estimated on the training data, and the model with the lowest \gls{bic} in each zone was selected for use. The selected models and their estimated parameters can be seen in \cref{tab:autocorrelation:parameters}

\begin{table}[htb]
\centering
\caption[Estimated Parameters]{Estimated parameters for the SARMA model. }
\label{tab:autocorrelation:parameters}
\begin{tabular}{lccccc}
\toprule
 &  & DK1-offshore & DK1-onshore & DK2-offshore & DK2-onshore \\
\midrule
$\sigma^2$ &  & $0.307 \pm 0.002$ & $0.122 \pm 0.001$ & $0.295 \pm 0.002$ & $0.17 \pm 0.001$ \\
\cline{1-6}
\multirow[c]{3}{*}{$\phi_i$} & 1 & $1.722 \pm 0.017$ & $1.203 \pm 0.007$ & $0.998 \pm 0.061$ & $0.838 \pm 0.005$ \\
 & 2 & $-0.731 \pm 0.015$ & $-0.377 \pm 0.01$ & $0.642 \pm 0.11$ & - \\
 & 3 & - & $0.055 \pm 0.007$ & $-0.642 \pm 0.05$ & - \\
\cline{1-6}
\multirow[c]{3}{*}{$\theta_i$} & 1 & $-0.734 \pm 0.018$ & - & $0.046 \pm 0.06$ & $0.256 \pm 0.006$ \\
 & 2 & $-0.188 \pm 0.007$ & - & $-0.816 \pm 0.05$ & - \\
 & 3 & - & - & $-0.205 \pm 0.012$ & - \\
\cline{1-6}
\multirow[c]{2}{*}{$\Phi_{24, i}$} & 1 & $1.006 \pm 0.007$ & $0.984 \pm 0.003$ & - & $0.995 \pm 0.001$ \\
 & 2 & $-0.006 \pm 0.007$ & - & - & - \\
\cline{1-6}
\multirow[c]{2}{*}{$\Theta_{24,i}$} & 1 & $-0.997 \pm 0.002$ & $-0.914 \pm 0.007$ & - & $-0.978 \pm 0.003$ \\
 & 2 & - & $-0.031 \pm 0.007$ & - & - \\
\cline{1-6}
\bottomrule
\end{tabular}
\end{table}


In DK2-offshore there was no seasonal terms selected, possibly related to the ensemble failures.

It can also be noted that the residual variance in the onshore zone are smaller than in the offshores zone. This would indicate that the onshore production is more predictable.

The models appeared to handle all autocorrelation present in the observations, as can be seen in the diagnostic plots, \cref{fig:autocorrelation:sarmaresiduals} for DK1 onshore and all zones in \cref{appedix:autocorrelation} with no significant lags in either \gls{acf} or \gls{pacf}  in any zone. 

\begin{figure}[htb]
    \centering
    \caption[SARMA diagnostic plots]{Diagnostic plots for the \gls{sarma}((3,0,0), (1,0,2,24) model in DK1-onshore}
    \includegraphics[width=1\linewidth]{Results/Autocorrelation/Figures/Residuals/DK1-onshore.pdf}
    \label{fig:autocorrelation:sarmaresiduals}
\end{figure}

The tails of the distribution still seem to be a problem. The histograms show more centred distributions than expected, while there are quite a few outliers in all zones.
